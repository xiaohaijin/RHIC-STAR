%$Id: urqmd-user.tex,v 1.1 2012/11/29 20:57:12 jwebb Exp $
\documentclass[12pt]{article}
\usepackage{psfig,rotating,times,courier,hyperref}
\renewcommand{\textheight}{22cm}
\textwidth17cm
\addtolength{\headheight}{-2cm}
\addtolength{\oddsidemargin}{-2cm}
%\addtolength{\evensidemargin}{-2cm}

\newcommand{\uqmd}{{\tt UrQMD}}
\newcommand{\uversion}{{\tt 3.3}}
%
\newcommand{\syn}[1]{\fbox{ \begin{minipage}{10cm} \tt #1 \end{minipage}}}
\newcommand{\des}[1]{\item{\tt #1}}
\pagestyle{plain}

\parindent=0pt
\sloppy
\title{The \raisebox{-0.9ex}{\psfig{figure=urqmd.eps,height=1.2cm}} user guide}
\author{}
\date{December, 9th 2009}
\begin{document}
\maketitle

\begin{center}
\fbox{
\begin{minipage}{12cm}
{\bf Warning:} This document is updated regularly. In its current form
it describes the handling of {\bf \uqmd{} revision \uversion}. If you are
using a different version of \uqmd, please obtain the manual from
{\tt \href{http://urqmd.org/documentation}{http://urqmd.org/documentation}}

The authors give no warranty to the correct functioning of the
\uqmd{} program. Use this program at your own risk. Please send all
bug--reports to the following e-mail address:
\vspace{0.15cm}
\begin{center}
{\tt \href{mailto:urqmd@urqmd.org}{urqmd@urqmd.org}} and a copy to\\
{\tt bleicher@th.physik.uni-frankfurt.de}
\end{center}
\vspace{0.15cm}
\end{minipage}}
\end{center}


\pagebreak

\section*{General Information}
The Ultra Relativistic Quantum Molecular Dynamics
 (\uqmd) model is a transport model for simulating heavy ion collisions in
the energy range from SIS to RHIC (use at LHC is at your own risk).
It runs on various UNIX-based computing platforms.
Current implementations include
IBM/AIX (xlf), GNU/Linux (gfortran, ifc), SGI/IRIX, DEC-UNIX
and Sun/Solaris.

\uqmd{} is designed as multipurpose tool  for studying a wide variety of
heavy ion related effects ranging from multifragmentation and collective
flow to particle production and correlations.
For hard pQCD scatterings, the model 
includes the PYTHIA routines from the LUND group.

This document is no introduction to the physics of \uqmd. Its purpose is
to serve as a short guide to the experienced physicist
on how to run the program. A detailed model description can be found in
the following two articles.
\begin{enumerate}
\item {\em Microscopic Models for Ultrarelativistic Heavy Ion Collisions}\\
        S.~A.~Bass, M.~Belkacem, M.~Bleicher, M.~Brandstetter, L.~Bravina, 
        C.~Ernst, L.~Gerland, M.~Hofmann, S.~Hofmann, J.~Konopka, G.~Mao, 
        L.~Neise, S.~Soff, C.~Spieles, H.~Weber, L.~A.~Winckelmann, 
        H.~St\"ocker, W.~Greiner, C.~Hartnack, J.~Aichelin and N.~Amelin.\\
        Prog.~Part.~Nucl.~Phys.~{\bf 41} (1998) 225--370.
\item{\em Relativistic Hadron-Hadron Collisions and the Ultra-Relativistic
        Quantum Molecular Dynamics Model (UrQMD)}\\
        M.~Bleicher, E.~Zabrodin, C.~Spieles, S.A.~Bass, C.~Ernst,
        S.~Soff, H.~Weber, H.~St\"ocker and W.~Greiner.\\
	J.~Phys.~{\bf G25} (1999), 1859--1896.
\end{enumerate}

For the hybrid model including an ideal hydrodynamic evolution for the hot and dense stage please refer to 
\begin{enumerate}
\item {\em  Fully integrated transport approach to heavy ion reactions with an intermediate hydrodynamic stage}\\
H.~Petersen, J.~Steinheimer, G.~Burau, M.~Bleicher and H.~St\"ocker.\\
   Phys.\ Rev.\  C {\bf 78} (2008) 044901.

\end{enumerate}
The hybrid calculations are only tested and give reasonable results in the energy range from $E_{\rm lab} = 2A -160A$ GeV. 


The hydrodynamic evolution is carried out by the Smooth and Sharp Transport Algorithm (SHASTA) in its implementation for relativistic
heavy ion collisions as it is decribed in 
\begin{enumerate}
\item {\em Relativistic hydrodynamics for heavy ion collisions. 1. General aspects and
  expansion into vacuum}\\
  D.~H.~Rischke, S.~Bernard and J.~A.~Maruhn. \\
  Nucl.~Phys.~A {\bf 595} (1995) 346.

\item {\em Relativistic hydrodynamics for heavy ion collisions. 2. Compression of
  nuclear matter and the phase transition to the quark - gluon plasma}\\
  D.~H.~Rischke, Y.~Pursun and J.~A.~Maruhn. \\
   Nucl.\ Phys.\  A {\bf 595} (1995) 383.
\end{enumerate}

For the different equations of state during the hydrodynamic evolution the reader is refered to the following references: 
\begin{enumerate}
\item Hadron Gas:\\
{\em Particle ratios at RHIC: Effective hadron masses and chemical freeze-out}\\
  D.~Zschiesche, S.~Schramm, J.~Schaffner-Bielich, H.~St\"ocker and W.~Greiner.\\
  Phys.\ Lett.\  B {\bf 547}, 7 (2002).

\item Bag Model:\\
{\em Relativistic hydrodynamics for heavy ion collisions. 2. Compression of nuclear matter and the phase transition to the quark - gluon plasma}\\
  D.~H.~Rischke, Y.~Pursun and J.~A.~Maruhn.\\
  Nucl.\ Phys.\  A {\bf 595}, 383 (1995).

\item Chiral EoS: \\
 {\em (3+1)-Dimensional Hydrodynamic Expansion with a Critical Point from
  Realistic Initial Conditions}\\
  J.~Steinheimer, M.~Bleicher, H.~Petersen, S.~Schramm, H.~St\"ocker and D.~Zschiesche.\\
  Phys.\ Rev.\  C {\bf 77}, 034901 (2008).

\end{enumerate}

\section*{User support}

\uqmd{} users are encouraged to join the user@urqmd.org e-mail list and post
questions concerning \uqmd{} to this list. When registering as \uqmd{} user at
\href{http://urqmd.org/}{http://urqmd.org/}, subscription to the list will
happen automatically.

For further questions and bug reports, the \uqmd{}-developers can be
contacted via \href{mailto:urqmd@urqmd.org}{urqmd@urqmd.org}

%\pagebreak

\section*{Copyright}
\uqmd{} source and documentation are provided freely for the purpose of
checking and reproducing published results of the authors. 

The Open Standard Codes and Routines (OSCAR)-Group has established - for
good reasons - guidelines for reproducablity, usage and quality control
of simlulations codes for pA and AA collisions.

\uqmd{} is a complex model. In order to ensure that it is used correctly, that all
results are reproducible and that the proper credits are given we ask for your
agreement to the following copyright and safeguard
mechanisms in the OSCAR spirit.

The \uqmd{} collaboration favors cooperation and joint projects with
outside researchers. We encourage experimental collaborations to compare
their results to \uqmd. We support you and/or cooperate on any
sensible project related to \uqmd.

If you are interested in a project, please contact us.

Projects without the participation of the \uqmd-Collaboration are accepted,
if the project is not a current thesis topic of any \uqmd-Collaboration
member.

We expect that the code authors are informed about any changes and
modifications made to the code. Any changes to the official version must be
documented.

{\bf The only official source for the \uqmd{} program} is the web 
page {\tt \href{http://urqmd.org}{http://urqmd.org}}\,.

The code or any fragments of it shall not be given away to third parties.
Similarily, events generated with \uqmd{} shall not be given to third parties
without consent of the code authors at Frankfurt.


\section*{Compiling and running the program}

To compile \uqmd{} one needs a FORTRAN77 compiler and GNU-make.  The
GNU-make programm is available on {\tt
\href{ftp://ftp.gnu.org}{ftp.gnu.org}} (note: on many old UNIX
systems GNU-make is called {\tt gmake}). Compilation is initiated by issuing
the {\tt make} command at the command-prompt in the \uqmd{} subdirectory.
After successful compilation the binary has the name {\tt urqmd.}ARCH where
ARCH is the machine type as given by {\tt uname -m}. For further
possibilities of using make with \uqmd{}, type {\tt (g)make help}.

In order to run \uqmd{} one needs to define the running parameters with an
input file.
The input file is made accessible to \uqmd{} by attaching its name to the environment-variable {\tt ftn09}. 
The output files 
are attached in the same fashion via the environment variables
{\tt ftn14} and {\tt ftn15}. Figure \ref{cccrun} shows, how
\uqmd{} is started on a generic UNIX system (here Linux using the Bash Shell). 
A sample file (runqmd.bash) is provided:
\begin{figure}[h]
\begin{center}
\fbox{
\begin{minipage}{13cm}
{\tt
export ftn09=inputfile \\
export ftn13=outputfile\_with\_freezeout\\
export ftn14=outputfile\\
export ftn15=collisionfile\\
export ftn16=outputfile\_with\_decaying\_resonances\\
export ftn19=outputfile\_for\_OSCAR97\\
export ftn20=outputfile\_for\_OSCAR99\\
urqmd.\$(uname -m)}

\end{minipage}}
\end{center}
\caption{\label{cccrun} running the \uqmd{} program}
\end{figure}

\pagebreak

\section*{The input file}

Figure \ref{cccinput} shows a typical input--file for \uqmd.
The general format of the inputfile is {\tt 1A3,1A77}. This means
that every input line consists of two sections: First a three character
flag followed by a 77 character string, the contents of which varies according
to the flag specified. A sample inputfile (inputfile) is included.

\begin{figure}[ht]
\begin{verbatim}
# this is a sample input file for urqmd
#  projectile
#    Ap Zp
pro  197 79
# optional: special projectile: ityp, iso3
# PRO 101 2 
#  target
#    At Zt
tar  197 79
#  number of events
nev 10
#  time to propagate and output time-interval (in fm/c)
tim  200 200
#
#   incident beam energy in AGeV
elb  160.0
#
#   weighted impact parameter distribution (from 0-3 fm)
imp  -3.0
#
#   equation of state
eos  0        # CASCADE mode
#   some options and parameters 
cto  4  1     # output of initialization
ctp  1  1.d0    # scaling for decay width of Resonances
#
f15  # no output to file15
#  end of file
xxx
\end{verbatim}
\caption{\label{cccinput} sample input file for \uqmd}
\end{figure}


The input--file does not have a predefined sequence. 
However, it is mandatory that the input contains definitions for projectile,
target, impact parameter and incident beam energy.

\begin{table}
\begin{center}
\renewcommand{\arraystretch}{1.2}
\begin{tabular}{|c|l|l|}\hline
label& arguments & description \\\hline\hline
\tt \#    &               & comment line \\
\tt xxx   &               & last line of input--file \\
\tt pro   & \tt Ap Zp     & define projectile \\
\tt PRO   & \tt ityp iso3 & define special projectile \\ 
\tt tar   & \tt At Zt     & define target \\
\tt TAR   & \tt ityp iso3 & define special target \\ 
\tt nev   & \tt nevents & number of events to calculate \\
\tt tim   & \tt tottime outtime  & define time of calculation and output \\
\tt ene   & \tt ebeam & incident kinetic beam energy (lab frame) \\
\tt elb   & \tt ebeam & incident kinetic beam energy (lab frame)\\
\tt plb   & \tt pbeam & incident beam momentum (lab frame)\\
\tt PLB   & \tt pmin pmax npbin & incident (min/max) beam momentum for 
			          excitation function\\
\tt PLG   & \tt pmin pmax npbin & like {\tt PLB}, log-weighted \\
\tt ecm   & \tt srt  & $\sqrt{s}$ for two particle collision\\
\tt ENE   & \tt srtmin srtmax nsrt & incident min/max $\sqrt{s}$ for 
				     excitation function \\
\tt ELG   & \tt srtmin srtmax nsrt & incident min/max $\sqrt{s}$ for 
				excitation function (log-weighted) \\
\tt imp   & \tt bmax & define impact parameter ({\tt bmin=0}) \\
\tt IMP   & \tt bmin bmax & define impact parameter \\
\tt eos   & \tt EoS & define equation of state\\
\tt box   & \tt dim edens solid para & define box for infinite matter calculation\\
\tt bpt   & \tt ityp  iso3  npart  pmax & define particle population for box-mode\\
\tt bpe   & \tt ityp  iso3  npart & like {\tt bpt}, for given energy density \\
\tt rsd   & \tt seed & seed for random number generator \\
\tt stb   & \tt ityp & keep particle stable \\
\tt cdt   & \tt deltat & $\Delta t$ between full collision load \\
\tt f13   & \tt  & suppress output to unit 13 \\
\tt f14   & \tt  & suppress output to unit 14 \\
\tt f15   & \tt  & suppress output to unit 15 \\
\tt f16   & \tt  & suppress output to unit 16 \\
%\tt f18   & \tt  & supress output to unit 18 \\
\tt f19   & \tt  & suppress output to unit 19 \\
\tt f20   & \tt  & suppress output to unit 20 \\
\tt ctp   & \tt index value &  set optional parameter in CTParam array \\
\tt cto   & \tt index value &  set option in CTOption array \\
\hline 
\end{tabular}
\end{center}

\caption{\label{inputcodes} possible flags in the input--file with their 
respective arguments}
\end{table}

Table \ref{inputcodes} shows a quick summary of all possible flags with their respective
parameters. 

\pagebreak

\subsection*{Input Parameters}
In this section all input labels with their respective arguments are explained. A
complete sample input file can be seen in figure \ref{cccinput}.

\subsubsection*{comment}
\syn{ \#~ string }
\begin{description}
\des{string} can be used to insert comments into the input file.
\end{description}

\subsubsection*{end of file}
\syn{ xxx string }
\begin{description}
\des{string} should contain at least one blank, {\tt xxx} marks the end of
 the input file. On some systems it might be necessary to add an additional empty line after
the {\tt xxx}.
\end{description}

\subsubsection*{define projectile}
\syn{ pro Ap Zp \\
      PRO ityp iso3 }
\begin{description}
\des{Ap} mass of projectile
\des{Zp} charge of projectile\\
\end{description}
Instead of defining an ordinary nucleus (with {\tt pro} one can also define
a special non-composite projectile with {\tt PRO}:\\
\begin{description}
\des{ityp} ID of projectile (see tables \ref{bitypes} and \ref{mitypes} for available itypes)
\des{iso3} $~~2\cdot Isospin_3$ of particle 
\end{description}

\subsubsection*{define target}
\syn{ tar Ap Zp\\
      TAR ityp iso3 }
\begin{description}
\des{Ap} mass of target
\des{Zp} charge of target\\
\end{description}
Instead of defining an ordinary nucleus (with {\tt tar} one can also define
a special non-composite target with {\tt TAR}:\\
\begin{description}
\des{ityp} ID of target (see tables \ref{bitypes} and \ref{mitypes} for available itypes)
\des{iso3} $~~2\cdot Isospin_3$ of particle
\end{description}

\subsubsection*{define number of events}
\syn{nev nevents}
\begin{description}
\des{nevents} number of events to calculate
\end{description}

\subsubsection*{define calculation time (in fm/c)}
\syn{tim tottime outtime}
\begin{description}
\des{tottime} total time span (in fm/c) to calculate
\des{outtime} time interval (in fm/c) after which output is written to 
files 13 and 14
\end{description}

\subsubsection*{define incident beam energy (in GeV)}
\syn{ene ebeam \\
     elb ebeam \\
     ecm srts \\
     ENE srtmin srtmax nsrt \\
     ELG srtmin srtmax nsrt \\
     plb pbeam\\
     PLB pmin pmax npbin\\
     PLG pmin pmax npbin} 
\begin{description}
\des{ebeam} kinetic energy of the beam-particle 
(in case of nuclei it is the energy per nucleon) in the laboratory frame
\des{srt} $\sqrt{s_{NN}}$ between projectile and target (in case of nuclei it is the energy per nucleon pair)
\des{srtmin} minimal value for $\sqrt{s_{NN}}$ between projectile and target
particles (in case of nuclei it is the energy per nucleon)
\des{srtmax} maximal value for $\sqrt{s_{NN}}$ between projectile and target
particles (in case of nuclei it is the energy per nucleon)
\des{nsrt} number of $\sqrt{s_{NN}}$ values from {\tt srtmin} to {\tt srtmax}
for which events shall be calculated (excitation function)
\des{pbeam} momentum of the beam-particle (in case of nuclei it is the momentum per nucleon)
\des{pmin} minimal value for $p_{lab}$ (for excitation functions)
\des{pmax} maximal value for $p_{lab}$ (for excitation functions)
\des{npbin} number of $p_{lab}$ values from {\tt pmin} to {\tt pmax}
for which events shall be calculated (excitation function)
\end{description}
For single momenta/energies the definitions {\tt ene}/{\tt elb, ecm} and 
{\tt plb} are used, for excitation functions {\tt ENE, ELG, PLB, PLG} are
needed. The binning of the excitation function is linear for {\tt ENE} and
{\tt PLB} and logarithmic for {\tt ELG} and {\tt PLG}. In the case
of an excitation function the number of events {\tt nev} refers to the
{\bf full} excitation function, i.e. the number of events per bin would
be {\tt nev}/{\tt nsrt} or {\tt nev}/{\tt npbin} respectively. 

Only one of the above seven definitions must be given. Make
sure to only use one of the above commands for the beam energy in order
to avoid ambiguities in the input file.


\subsubsection*{define impact parameter}
\syn{imp b \\
     imp -bmax\\
     IMP bmin bmax}
\begin{description}
\des{b} fixed impact parameter of $b$ fm.
\des{-bmax} impact parameter range from $0\dots b_{max}$.
\des{bmin} minimum impact parameter.
\des{bmax} maximum impact parameter.
\end{description}

By default, the impact parameter is weighted quadratically ({\tt CTOption(5)} is automatically set). 
However, with {\tt CTOption(5)} it is possible to change the weighting characteristic also to 
a linear weighting (this is in contrast to the usual experimental trigger conditions).

A minimum bias calculation (including events without interaction!) can be performed with {\tt bmin=0},{\tt bmax > }
$R_p+R_t$. 

\subsubsection*{infinite matter (box) calculations}
\syn{ box dim edens solid para \\
      bpt ityp iso3  npart pmax \\
      bpe ityp iso3  npart} 
\begin{description}
\des{dim} width of (cubic) box
\des{edens} total energy content of box in GeV
\des{solid} 1: reflecting walls, 0: periodic boundary conditions
\des{para} 0: standard, 1: use ``old'' periodic boundary conditions
\des{ityp} ID of species (see tables \ref{bitypes} and \ref{mitypes} for available itypes)
\des{iso3} $~~2\cdot Isospin_3$ of species
\des{npart} number of particles for species
\des{pmax} maximum momentum for fermi-sphere in momentum space.
\end{description}


\subsubsection*{define equation of state}
\syn{eos EoS}
\begin{description}
\des{EoS} equation of state for the calculation 
%(see table \ref{eostab}).
\end{description}
Currently only CASCADE mode ({\tt EoS=0}) or a hard Skyrme equation of state
({\tt EoS=1}) are available. The default mode is CASCADE, the hard Skyrme equation
of state is limited to incident beam-energies below 4.0 GeV/nucleon.

{\bf Important:} This option also changes the initialization mode 
(see {\tt CTOption(24)}).

This option has nothing to do with the equation of state during the hydrodynamic evolution in the hybrid mode. 
\subsubsection*{set random number generator seed}
\syn{rsd seed}
\begin{description}
\des{seed} (integer) seed for random number generator
\end{description}
On many computer systems, \uqmd{} is able to extract a random 
seed from the local time at intervals of one second.
However, we advise to check if this is indeed the case for your system.
Especially, if running many \uqmd{} jobs in parallel it is mandatory to set different
individual seeds to avoid synchronisation of the runs due to the same start time.

\subsubsection*{set forced collision load update interval }
\syn{cdt deltat}
\begin{description}
\des{deltat} time interval (in fm/c) for the update of potentials and
a full particle scan for the collision arrays
\end{description}

In the CASCADE mode a regular full particle scan for the collision arrays 
is not necessary, therefore this command should be only used in
calculations including potentials, infinite matter calculations  
or for debug purposes.

\subsubsection*{suppress output files}
\syn{f13 \\
     f14 \\
     f15 \\
     f16 \\
%     f18 \\
     f19\\
     f20}

\vspace{.5cm}
The output to the respective files is omitted if the above command is
used.

\subsubsection*{set particles stable}
\syn{stb ityp}
\begin{description}
\des{ityp} ID of particle (see tables \ref{bitypes} and \ref{mitypes} for available itypes)
\end{description}

Treat all particles with this ID as stable particles. At the moment the number of particles to be set stable is limited to
20 via the parameter {\tt maxstables} in {\tt options.f}.

\subsubsection*{set special parameter}
\syn{ctp index value}
\begin{description}
\des{index} index of {\tt CTParam()} array
\des{value} value for {\tt CTParam(index)} 
	(see tables \ref{ctparams1} and \ref{ctparams2} for available parameters)
\end{description}

\subsubsection*{set special option}
\syn{cto index value}
\begin{description}
\des{index} index of {\tt CTOption()} array
\des{value} value for {\tt CTOption(index)}
	(see table \ref{ctoptions} for available options)
\end{description}


Please note that if you are running \uqmd from a self-generated old inputfile via {\tt CTOption(40)=1} make sure that all the baryons are listed first followed by all the mesons. This particle order is necessary to avoid further complications. 

If {\tt CTOption(45)=1} is used the code needs 2GB of memory because of the dimensions of the hydrodynamic grid and the corresponding array sizes. 

\begin{table}[h]
\begin{center}
\renewcommand{\arraystretch}{1.3}
\begin{tabular}{|rc|rc|rc|rc|rc|rc|}
\hline 
\bf ID &\bf nucleon & \bf ID & \bf delta &\bf ID &\bf lambda &\bf ID &\bf sigma &\bf ID &\bf xi &\bf ID &\bf omega\\  \hline \hline
1 & $N_{938}$& 17 &$\Delta_{1232}$& 27 &$\Lambda_{1116}$& 40 &$\Sigma_{1192}$& 49 &$\Xi_{1317}$& 55 &$\Omega_{1672}$\\
2 &$N_{1440}$& 18 &$\Delta_{1600}$& 28 &$\Lambda_{1405}$& 41 &$\Sigma_{1385}$& 50 &$\Xi_{1530}$&    &\\
3 &$N_{1520}$& 19 &$\Delta_{1620}$& 29 &$\Lambda_{1520}$& 42 &$\Sigma_{1660}$& 51 &$\Xi_{1690}$&    &\\
4 &$N_{1535}$& 20 &$\Delta_{1700}$& 30 &$\Lambda_{1600}$& 43 &$\Sigma_{1670}$& 52 &$\Xi_{1820}$&    &\\
5 &$N_{1650}$& 21 &$\Delta_{1900}$& 31 &$\Lambda_{1670}$& 44 &$\Sigma_{1775}$& 53 &$\Xi_{1950}$&    &\\
6 &$N_{1675}$& 22 &$\Delta_{1905}$& 32 &$\Lambda_{1690}$& 45 &$\Sigma_{1790}$& 54 &$\Xi_{2025}$&    &\\
7 &$N_{1680}$& 23 &$\Delta_{1910}$& 33 &$\Lambda_{1800}$& 46 &$\Sigma_{1915}$&    &            &    &\\
8 &$N_{1700}$& 24 &$\Delta_{1920}$& 34 &$\Lambda_{1810}$& 47 &$\Sigma_{1940}$&    &            &    &\\
9 &$N_{1710}$& 25 &$\Delta_{1930}$& 35 &$\Lambda_{1820}$& 48 &$\Sigma_{2030}$&    &            &    &\\
10&$N_{1720}$& 26 &$\Delta_{1950}$& 36 &$\Lambda_{1830}$&    &               &    &            &    &\\
11&$N_{1900}$&    &               & 37 &$\Lambda_{1890}$&    &               &    &            &    &\\
12&$N_{1990}$&    &               & 38 &$\Lambda_{2100}$&    &               &    &            &    &\\
13&$N_{2080}$&    &               & 39 &$\Lambda_{2110}$&    &               &    &            &    &\\
14&$N_{2190}$&    &               &    &                &    &               &    &            &    &\\
15&$N_{2200}$&    &               &    &                &    &               &    &            &    &\\
16&$N_{2250}$&    &               &    &                &    &               &    &            &    &\\
\hline
\end{tabular}
\caption{\label{bitypes} Baryon-ID's used in \uqmd. A particle is fully defined,
when its ityp and  $2\cdot I_3$ are known. Antibaryons carry a negative sign.}
\vspace{0.5cm}
\begin{tabular}{|rc|rc|rc|rc|}
\hline 
\bf ID &$0^{-+}$ & \bf ID & $1^{--}$ & \bf ID & $ 0^{++}$ & \bf ID & $ 1^{++}$ \\ \hline \hline 
101 & $\pi$     & 104 &$ \rho$     & 111 & $ a_0$      & 114   & $ a_1$    \\
106 & $K  $     & 108 &$   K^*$    & 110 & $ K_0^*$    & 113   & $ K_1^*$  \\
102 & $\eta$    & 103 &$  \omega$  & 105 & $ f_0 $     & 115   &  $f_1$    \\
107 & $\eta'$   & 109 &$\phi $     & 112 & $f_0^*$     & 116   & $ f_1'$   \\
\hline\hline
\bf ID & $1^{+-}$ & \bf ID &$ 2^{++}$ & \bf ID & $(1^{--})^*$  & \bf ID & $(1^{--})^{**}$\\ \hline \hline
122 & $ b_1 $    & 118 & $ a_2$     & 126 & $ \rho_{1450}$  & 130   & $\rho_{1700}$\\ 
121 & $ K_1 $    & 117 &$ K_2^*$    & 125 & $ K^*_{1410}$   & 129   & $K^*_{1680}$\\
123 & $ h_1 $    & 119 & $ f_2 $    & 127 & $ \omega_{1420}$& 131   & $\omega_{1662}$\\
124 & $ h_1'$    & 120 & $ f_2'$    & 128 & $ \phi_{1680}$  & 132   & $\phi_{1900}$\\\hline
\end{tabular}
\caption{\label{mitypes} Meson-ID's in \uqmd, sorted with
respect to spin and parity, included into
the \uqmd{} model. Mesons with strangeness -1 carry a negative sign. }
\end{center}
\end{table}


%\begin{table}[h]
%\begin{center}
%\renewcommand{\arraystretch}{1.2}
%\begin{tabular}{|l|l|}\hline
%\bf \tt eos &   \\\hline\hline
%0  & CASCADE mode \\
%1  & \\ 
%2  &  \\\hline
%\end{tabular}
%\end{center}
%\caption{\label{eostab}  available interactions for \uqmd}
%\end{table}


\begin{table}[h]
\begin{center}
\renewcommand{\arraystretch}{1.2}
\begin{tabular}{|l|l|l|}\hline
\bf CTParam(X) & default & function  \\\hline\hline
1  & 1.d0   & scaling factor for resonance widths \\
2  & 0.52d0 & minimal stringmass and el/inel. cut in {\tt makestr} \\ 
3  & 2.0d0  & velocity exponent for modified AQM   \\
4  & 0.3d0  & transverse pion mass, used in {\tt strexct} and {\tt make22} \\
5  & 0.0d0  & probability for quark rearrangement in cluster \\
6  & 0.37d0 & strangeness probability in {\tt makestr} \\
7  & 0.d0   & charm probability (not yet implemented in \uqmd) \\
8  & 0.093d0& probability to create a diquark \\
9  & 0.35d0 & kinetic energy cut-off for last string break\\
10 & 0.25d0 & min. kinetic energy for hadron (in string) \\
11 & 0.0d0  & percentage of non groundstate resonances (in string) \\
%12 & 0.5d0  & probability for $\rho_{770}$ (in string) \\
%13 & 0.27d0 & probability for $\rho_{1450}$ (rest in $\rho_{1700}$) in string \\
%14 & 0.49d0 & probability for $\omega_{782}$ in string \\
%15 & 0.27d0 & probability for $\omega_{1420}$ (rest in $\omega_{1600}$) in string \\
%16 & 1.0d0  & mass cut betw. $\rho_{770}$ and $\rho_{1450}$\\
%17 & 1.6d0  & mass cut betw. $\rho_{1450}$ and $\rho_{1700}$ \\
%18 & 0.85d0 & mass cut betw. $\omega_{782}$ and $\omega_{1420}$ \\
%19 & 1.55d0 & mass cut betw. $\omega_{1420}$ and $\omega_{1600}$ \\
%20 & 0.17d0 & probability for 0- mesons \\
%21 & 0.28d0 & probability for 1- mesons \\
21 & 0.d0 & deformation parameter \\
25 & 0.9d0  & probability for diquark not to break \\
%26 & 5.d0   & maximum trials to get string masses\\
%27 & 1.d0   & scaling factor for {\tt xmin} in string excitation\\
28 & 1.d0   & scaling factor for transverse fermi motion\\
29 & 1.d0  & double strange di-quark suppression factor \\
30 & 1.5d0  & radius offset for initialization \\
31 & 1.6d0  & $\sigma$ of Gaussian for transverse momentum transfer\\ 
32 & 0.d0   & $\alpha-1$ for valence quark distribution\\
33 & 2.5d0  & $\beta_v$ for valence quark distribution\\ 
34 & 0.1d0  & minimal $x$ multiplied with $E_{c.m.}$ \\
35 & 3.0d0  & offset for cut for the FSM \\
36 & 0.275d0& fragmentation function parameter $a$ (nucleons) \\
37 & 0.42d0 & fragmentation function parameter $b$ (nucleons) \\
38 & 1.08d0 & diquark $p_t$ scaling factor \\
39 & 0.8d0  & strange quark $p_t$ scaling factor \\
40 & 0.5d0  & $\beta_s-1$ for valence quark distribution \\
41 & 0.d0   & distance between nuclei at initialization \\
42 & 0.55d0 & width of Gaussian for $p_t$-distribution in string-fragmentation \\
43 & 5.0d0  & maximum kinetic energy in mesonic cluster\\ 
%44 & 0.8d0  & probability of double vs. single excitation in AQM inelastic\\
%45 & 0.5d0  & offset for minimal mass generation of strings\\
46 & 8.0d6  & maximum number of rejections during initialization of nuclei\\
47 & 1.0d0  & Field-Feynman fragmentation func. parameter $a$ (prod. part.)\\
48 & 2.0d0  & Field-Feynman fragmentation func. parameter $b$ (prod. part.)\\
\hline
\end{tabular}
\caption{\label{ctparams1}  Optional parameters used in \uqmd}
\end{center}
\end{table}

\begin{table}[h]
\begin{center}
\renewcommand{\arraystretch}{1.2}
\begin{tabular}{|l|l|l|}\hline
\bf CTParam(X) & default & function  \\\hline\hline
49 & 0.5d0  & additional single strange diquark suppression factor \\
50 & 1.0d0  & enhancement factor for $0^{-+}$ mesons \\
51 & 1.0d0  & enhancement factor for $1^{--}$ mesons \\
52 & 1.0d0  & enhancement factor for $0^{++}$ mesons \\
53 & 1.0d0  & enhancement factor for $1^{++}$ mesons \\
54 & 1.0d0  & enhancement factor for $2^{++}$ mesons \\
55 & 1.0d0  & enhancement factor for $1^{+-}$ mesons \\
56 & 1.0d0  & enhancement factor for $(1^{--})^*$ mesons \\
57 & 1.0d0  & enhancement factor for $(1^{--})^{**}$ mesons \\
58 & 1.0d0  & scaling factor for DPF time-delay \\
59 & 0.4d0  & scaling factor for leading hadron cross-section (PYTHIA)\\
60 & 3.0d0  & resonance/string transition energy for s-channel \\
61 & 0.2d0  & cell size $dx$ of the hydro code\\
62 & 200    & {\tt ngr} is the grid size of the hydro code\\
63 & 1.0d0  & minimum $t_{\rm start}$ for hydro calculation\\
64 & 5.0d0  & multiplied with $\epsilon_0$ as freeze-out criterion\\
65 & 1.0d0  & factor to be multiplied with $t_{\rm start}$\\
66 & 1.d10  & rapidity cut for hydrodynamic decription\\
67 & 1.d0   & integer number of testparticles per real particle\\
\hline
\end{tabular}
\caption{\label{ctparams2}  Optional parameters used in \uqmd}
\end{center}
\end{table}

\begin{table}[h]
\begin{center}
\renewcommand{\arraystretch}{1.2}
\begin{tabular}{|l|l|l|}\hline
\bf CTOption(X) & \bf default / & \bf description  \\
                & \bf options   &       \\\hline\hline
1  & 0 & mass dependent resonance decay widths   \\\hline
   & 0 & enabled \\\hline
   & 1 & disabled \\\hline\hline
2  & 0 &  2-particle scattering plane:  \\\hline
   & 0 &  stochastic selection of $\varphi(1,2)$ \\
   & 1 &  conserve plane \\\hline\hline
3  & 0 & detailed balance selection  \\\hline
   & 0 & take finite resonance widths into account \\
   & 1 & use standard detailed balance \\\hline\hline
4  & 0 & initial configuration output to file14 \\\hline
   & 0 & output according to {\tt tim} statement \\
   & 1 & {\bf additional } output of initialization \\\hline\hline
5  & 0 & impact parameter weighting \\\hline
   & 0 & use {\tt bmax} as fixed impact parameter \\
   & 1 & random $b$ from {\tt bmin} to {\tt bmax}, $bdb$ weighted \\
   & 2 & random $b$ from {\tt bmin} to {\tt bmax}, 
		flat distribution \\\hline\hline
6  & 0 & first collisions within target/projectile \\\hline
   & 0 & block first collisions {\bf within} proj./target \\
   & 1 & all collisions allowed \\\hline\hline
7  & 0 & suppress elastic $NN$ collisions \\ \hline
   & 0 & elastic collisions are allowed \\
   & 1 & no elastic $NN$ collisions; $\sigma_{in}=\sigma_{tot}$\\\hline\hline 
8  & 0 & mass dependent partial decay widths \\ \hline
   & 0 & enabled\\
   & 1 & disabled, use fixed widths \\ \hline\hline
9  & 0 & tabulated p+p inelastic cross sections \\\hline
   & 0 & enable table-lookup\\ 
   & 1 & disable table-lookup \\\hline\hline
10 & 0 & Pauli-blocker \\\hline
   & 0 & enable Pauli-blocker \\
   & 1 & disable Pauli-blocker \\\hline
\end{tabular}
\end{center}
\end{table}

\begin{table}[h]
\begin{center}
\renewcommand{\arraystretch}{1.2}
\begin{tabular}{|l|l|l|}\hline
\bf CTOption(X) & \bf default / & \bf description  \\
                & \bf options   &       \\\hline\hline
11 & 0 & mass reduction (binding energy) in CASCADE mode \\\hline
   & 0 & enable mass reduction according to binding energy\\
   & 1 & disable mass reduction  \\\hline\hline
12 & 0 & string production \\\hline
   & 0 & enable string production \\
   & 1 & disable string production \\\hline\hline
13 & 0 & enhanced file16 output \\\hline
   & 0 & disabled \\
   & 1 & enabled \\\hline\hline
14 & 0 & angular distribution in binary scattering \\\hline
   & 0 & enable angular distribution \\
   & 1 & disable distribution ($\cos(\vartheta)=1$ forward peak) \\\hline\hline
15 & 0 & meson-meson and meson-baryon scattering \\\hline
   & 0 & enable MM and MB scattering \\
   & 1 & disable MM and MB scattering \\\hline\hline
16 & 0 & molecular dynamics switch \\\hline
   & 0 & enable collision term \\
   & 1 & propagate with forces only (disable collision term) \\\hline\hline
17 & 0 & collision-table update mode \\\hline
   & 0 & update only collision partners after interaction \\
   & 1 & initialize complete table after every interaction \\\hline\hline
18 & 0 & decay of unstable particles at end of event \\\hline
   & 0 & perform decay after final timestep \\
   & 1 & unstable particles do not decay after final timestep \\\hline\hline
19 & 0 & $B \bar{B}$ annihilation \\\hline
   & 0 & enabled \\
   & 1 & disabled \\\hline\hline
20 & 0 & $e^+ e^-$ annihilation instead of $B \bar{B}$ annihilation \\\hline 
   & 0 & disabled (normal $\bar B B$ mode) \\
   & 1 & enabled  ($e^+ e^-$ mode) \\\hline
\end{tabular}
\end{center}
\end{table}

\begin{table}[h]
\begin{center}
\renewcommand{\arraystretch}{1.2}
\begin{tabular}{|l|l|l|}\hline
\bf CTOption(X) & \bf default / & \bf description  \\
                & \bf options   &       \\\hline\hline
21 & 0 & string fragmentation function \\\hline
   & 0 & field-Feynman  fragmentation function \\
   & 1 & Lund fragmentation function \\
   & 2 & QGSM fragmentation function \\\hline\hline
22 & 1 & string mass excitation \\\hline
   & -1& simple $1/M$ excitation \\
   & 1 & FRITIOF ansatz \\
   & 2 & QGSM ansatz \\\hline\hline
23 & 0 & Lorenz contraction of projectile and target \\\hline
   & 0 & enabled \\
   & 1 & disabled \\\hline\hline
24 & 1 & initialization mode \\\hline
   & 0 & hard sphere (used for EOS$\neq$0)\\
   & 1 & Woods-Saxon (used for CASCADE mode)\\ 
   & 2 & Fast Woods-Saxon (used for CASCADE mode)\\ \hline\hline
25 & 0 & phase space correction for resonance masses \\\hline
   & 0 & disabled \\
   & 1 & enabled \\\hline\hline
27 & 0 & reference frame for calculation \\\hline
   & 0 & N.N. (equal speed) frame \\
   & 1 & target (lab) frame \\
   & 2 & projectile frame \\\hline\hline
%28 & 0 & participant spectator model \\\hline
%   & 0 & full calculation \\
%   & 1 & interaction and propagation of participant nucleons only \\
%   & 2 & interaction of participants only, full propagation \\\hline\hline
29 & 2 & $p_t$ for last two particles in string 
	(\tt clustr)  \\\hline
   & 0 & isotropic  \\
   & 1 & isotropic, baryon goes into forward hemisphere \\ 
   & 2 & baryon goes into forward hemisphere, $p_t=0$ \\\hline\hline 
30 & 1 & frozen Fermi approximation in CASCADE mode \\\hline
   & 0 & disabled \\
   & 1 & enabled \\\hline\hline
32 & 0 & distribute resonance masses according to mass-dep. Breit-Wigner \\\hline
   & 0 & enabled \\
   & 1 & disabled \\\hline\hline
\end{tabular}
\end{center}
\end{table}


\begin{table}[h]
\begin{center}
\renewcommand{\arraystretch}{1.2}
\begin{tabular}{|l|l|l|}\hline
\bf CTOption(X) & \bf default / & \bf description  \\
                & \bf options   &       \\\hline\hline
33 & 0 & use table-lookup for calculation of $\langle p_{CMS}\rangle$ in {\tt pmean} \\\hline
   & 0 & enabled \\
   & 1 & disabled \\\hline\hline
34 & 1 & resonance life-times \\\hline
   & 0 & $\tau = 1/\Gamma(M)$ \\
   & 1 & $\tau = 1/\Gamma_{pole}$ \\
   & 2 & DPF formalism \\\hline\hline
35 & 1 & generate high-precision tables (file {\tt tables.dat}) \\\hline
   & 0 & disabled \\
   & 1 & enabled \\\hline\hline
36 & 0 & correct normalization for mass-dependent Breit-Wigner distributions \\\hline
   & 0 & enabled \\
   & 1 & disabled \\\hline\hline
37 & 0 & heavy quark clusters\\\hline
   & 0 & disabled \\
   & 1 & enabled \\\hline\hline
38 & 0 & scale p-pbar to b-bbar with equal p\_lab instead of equal $\sqrt s$\\\hline
   & 0 & disabled \\
   & 1 & enabled \\\hline\hline
39 & 0 & compute collision densities via call to Pauli-blocker \\\hline
   & 0 & enabled \\
   & 1 & disabled \\\hline\hline
40 & 0 & use old file14 as initial state for calculation \\\hline
   & 0 & disabled \\
   & 1 & enabled \\\hline\hline
41 & 0 & extended file14 output (needed for {\tt cto(40)}) \\\hline
   & 0 & disabled \\
   & 1 & enabled \\
   & 2 & different counting rules for {\tt origin} \\\hline\hline
42 & 0 & color fluctuations in high energy hadron-hadron collisions \\\hline
   & 0 & disabled \\
   & 1 & enabled \\\hline\hline
%43 & 0 & type of dilepton output \\\hline
%   & 0 & di-electron output \\
%   & 1 & di-muon output \\\hline\hline

\end{tabular}
\end{center}
\end{table}

\clearpage


\begin{table}[h]
\begin{center}
\renewcommand{\arraystretch}{1.2}
\begin{tabular}{|l|l|l|}\hline
\bf CTOption(X) & \bf default / & \bf description  \\
                & \bf options   &       \\\hline\hline
44 & 1 & Pythia call for hard scatterings \\\hline
   & 0 & disabled \\
   & 1 & enabled \\\hline\hline
45 & 0 & Hydro mode\\\hline
   & 0 & disabled \\
   & 1 & enabled \\\hline\hline
46 & 0 & Density calculation switch \\\hline
   & 0 & $\rho_{B}$ \\
   & 1 & $\rho_{q+\bar{q}}$ \\\hline\hline
47 & 2 & EoS for hydro evolution\\\hline
   & 2 & hadron gas (HG)\\
   & 3 & bag model (BM)\\
   & 5 & chiral+hadron gas (CH)\\\hline\hline
48 & 0 & number of timesteps for hydro propagation\\\hline
   & 0 & usual run until freeze-out \\
   & N & $N$ timesteps \\\hline\hline
49 & 0 & Spectator switch\\\hline
   & 0 & spectators are propagated in \uqmd \\
   & 1 & spectators are propagated on the hydro grid \\\hline\hline
50 & 0 & (Additional) f14/f19-output directly after the hydro to transport transition \\\hline
   & 0 & disabled \\
   & 1 & enabled \\\hline\hline
52 & 0 & Freeze-out switch for hydro mode\\\hline
   & 0 & Gradual transition scenario (GF) \\
   & 1 & Isochronuous transition scenario \\\hline\hline
53 & 0 & Improved momentum generation in Cooper-Frye\\\hline
   & 0 & enabled \\
   & 1 & disabled \\\hline\hline
\end{tabular}
\end{center}
\caption{\label{ctoptions} available options  in \uqmd}
\end{table}

\clearpage
\section*{Output files}
\label{output}

The \uqmd{} program has several different output files. The {\em standard output
files} ({\tt file13} and {\tt file14}) 
contain all particles of a given event at a certain time-step. 
The {\em collision history file} ({\tt file15}) 
contains information on all collisions/decays
of a given event. 
The {\em decay file} ({\tt file16}) contains information 
on all particle decays as well
as information on all stable particles after the final timestep. 
The {\em OSC files} ({\tt file19, file20}) generates output compliant with
the Open Standards And Codes (OSCAR) format.
Consecutive timesteps (only {\tt file13} and {\tt file14}) 
and events are added sequentially to the files.

Each event consists of a header and a body. The standard header
is identical for {\tt file13, file14} and {\tt file16} and {\tt file15}
use an abbreviated header and the format of 
{\tt file19} and {\tt file20} is fixed by the OSCAR requirements. 

\subsection*{Standard output-files: file13/file14}

The standard output files contain the phase-space of the event
at a given timestep (e.g. final output after last timestep).
{\tt file13} contains the same information as {\tt file14}, but
additionally lists the freeze-out coordinates in configuration-
and momentum-space for all particles.

Figure \ref{headerfig} shows  a standard header as used in
{\tt file13, file14} and {\tt file16}.
Only one header per event is written to file. Consecutive time-steps
of the same event are added body to body without additional headers
between them. In the case of running \uqmd{} in box mode, {\tt file14}
contains additional header-lines reporting the box-related parameters. 


The general format of the standard-fileheader can be found in 
table \ref{headerformfig}, its contents is self-explanatory -- please consult
figure~\ref{headerfig}.
\begin{table}
\begin{center}
\renewcommand{\arraystretch}{1.2}
\begin{tabular}{|l|l|}\hline
\bf line\# & \bf format \\\hline\hline
1       & \tt format(a20,3i7,a15,i2) \\
2       & \tt format(a13,a13,i4,i4,a12,a13,i4,i4,a1) \\
3	& \tt format(a36,3f11.7)\\
4	& \tt format(a36,3f6.2,a31,1f9.2) \\
5	& \tt format(a20,i3,a15,e11.4,a15,e11.4,a15,e11.4) \\
6	& \tt format(a7,i9,a13,i12,a9,a20,i7,a20,f11.3) \\
7	& \tt format(a2,15(i3,a2)) \\
8	& \tt format(a2,15(i3,a2)) \\
9	& \tt format(a2,15(i3,a2)) \\
10	& \tt format(a2,12(e11.4,a2)) \\
11	& \tt format(a2,12(e11.4,a2)) \\
12	& \tt format(a2,12(e11.4,a2)) \\
13	& \tt format(a2,12(e11.4,a2)) \\
14	& \tt format(a171)\\\hline
\end{tabular}
\end{center}
\caption{\label{headerformfig} format for the standard event header.  }
\end{table}

The format of the box-header can be found in table \ref{boxheadformfig},
please consult also figure~\ref{boxheaderfig}.
\begin{table}
\begin{center}
\renewcommand{\arraystretch}{1.2}
\begin{tabular}{|l|l|}\hline
\bf line\# & \bf format \\\hline\hline
1       & \tt format(a20,e14.6,a20,e14.6,a3,i1,a3,i1,a3,i3) \\
2       & \tt format(a35) \\
3ff	& \tt format(a5,2i4,i8,e14.6) \\ \hline
\end{tabular}
\end{center}
\caption{\label{boxheadformfig} format for the box-header extension.  }
\end{table}

All lines of the box-header are guaranteed to start with {\tt box}. The
first line contains the word {\tt boxmode}, followed by length of the box,
total energy and the parameters solid and para (see table~\ref{inputcodes})
and the number of specified particle species {\tt npart}. The second line contains no
physical information. The rest of the box-header contains npart lines, each
of them stating ityp, isospin, number and maximal momentum of the particles
as specified in the {\tt bpt} and {\tt bpe} input file directives.

The body of the  {\em standard output files} contains in its first line
the number of particles $N_{part}$ to follow (there are as many lines to follow
as there are particles) and the time (in fm/c) of the
output (two unformatted integers). 
The next line contains counters for the number of collisions, decays and
produced resonances per event ({\tt format(8i8)}), it is described in 
table \ref{countertab}.
\begin{table}
\begin{center}
\renewcommand{\arraystretch}{1.2}
\begin{tabular}{|l|l|}\hline
\bf column\# & \bf contents \\\hline\hline
1  & \# of collisions \\
2  & \# of elastic collisions \\
3  & \# of inelastic collisions \\
4  & \# of Pauli--blocked collisions \\
5  & \# of decays \\
6  & \# of produced {\em hard} baryon resonances \\
7  & \# of produced {\em soft} baryon resonances \\
8  & \# of baryon resonances produced via a decay of another resonance \\\hline
\end{tabular}
\end{center}
\caption{\label{countertab} description of the collision/decay counters in the
standard output file }
\end{table}

The subsequent $N_{part}$ lines then contain the information on the
individual particles. The exact format of the particle vector depends
on the chosen output file and the selected options. Table~\ref{stdvector}
lists the different possibilities.
Figure \ref{eventbodyfig} shows the beginning of
a sample event body for {\tt file14}.

\begin{table}
\begin{center}
\renewcommand{\arraystretch}{1.2}
\begin{tabular}{|l|l|}\hline
\bf format &  \\\hline\hline
\tt format(9e16.8,i11,2i3,i9,i5,i4)  & standard {\tt file14} and {\tt file16} \\
\tt format(9e16.8,i11,2i3,i9,i5,i10,3e16.8,i8) & {\tt file14} with {\tt CTOption(41)=1} \\
%\tt format(9e16.8,i7,2i3,i6,i5,i4,i5,2e16.8) &  {\tt file14} with {\tt CTOption(25)=1} \\
\tt format(9e16.8,i11,2i3,i9,i5,i4,8e16.8) & standard {\tt file13} \\
\tt format(9e15.7,i11,2i3,i9,i5,i4,2i4) & {\tt file16} with {\tt CTOption(13)}\\\hline
\end{tabular}
\end{center}
\caption{\label{stdvector} particle-vector format for different output-options
in the standard output files.}
\end{table}

The standard output files should suffice for most types of analysis. They
provide the event information at a given timestep (mostly the final
timestep).
The contents of the particle vectors is described in table
\ref{bodytab}. All reference frame dependent values are given in the 
computational frame, which has been fixed by {\tt CTOption(27)}.
\begin{table}
\begin{center}
\renewcommand{\arraystretch}{1.2}
\begin{tabular}{|l|l|}\hline
\bf column\# & \bf contents \\\hline\hline
1       & $t$   : eigentime of particle in fm/c \\
2       & $r_x$ : x coordinate in fm \\
3       & $r_y$ : y coordinate in fm \\
4       & $r_z$ : z coordinate in fm \\
5       & $E$   : energy of particle in GeV \\
6       & $p_x$ : x momentum component in GeV \\
7       & $p_y$ : y momentum component in GeV \\
8       & $p_z$ : z momentum component in GeV \\
9       & $m$ : mass of particle in GeV \\
10      & $ityp$ : particle-ID \\
11      & $2\cdot I_3$ : isospin z-projection (doubled) \\
12      & $ch$ : charge of particle \\
13      & index of last collision partner \\
14      & $N_{coll}$ number of collisions \\
15	& history information (parent process) \\\hline 
16       & $t_{fr}$   : freeze-out time of particle in fm/c ({\tt file13} only) \\
17       & $r_{x,fr}$ : freeze-out x coordinate in fm ({\tt file13} only) \\
18       & $r_{y,fr}$ : freeze-out y coordinate in fm ({\tt file13} only) \\
19       & $r_{z,fr}$ : freeze-out z coordinate in fm ({\tt file13} only) \\
20       & $E_{fr}$   : freeze-out energy of particle in GeV ({\tt file13} only) \\
21       & $p_{x,fr}$ : freeze-out x momentum component in GeV ({\tt file13} only) \\
22       & $p_{y,fr}$ : freeze-out y momentum component in GeV ({\tt file13} only) \\
23       & $p_{z,fr}$ : freeze-out z momentum component in GeV ({\tt file13} only) \\\hline
16	& $\tau_{dec}$ decay time of particle ({\tt file14} with {\tt CTOption(41)=1}) \\
17	& $\tau_{form}$ formation time of particle ({\tt file14} with {\tt CTOption(41)=1}) \\
18	& $R_{\sigma}$ cross section reduction factor ({\tt file14} with {\tt CTOption(41)=1}) \\
19      & unique particle number (not ID!)  ({\tt file14} with {\tt CTOption(41) =1}) \\\hline
16      & $ityp_{old,1}$ : particle-ID of parent particle \# 1 ({\tt file16} with {\tt CTOption(13)=1}) \\
17      & $ityp_{old,2}$ : particle-ID of parent particle \# 2 ({\tt file16} with {\tt CTOption(13)=1}) \\\hline 
\end{tabular}
\end{center}
\caption{\label{bodytab} contents of the particle vector in the
standard output files }
\end{table}


\subsection*{Collision history file: file15}

The collision file {\tt file15} contains each binary interaction,
resonance decay and string-excitation which occurred in the course
of the heavy-ion reaction. It can be used to reconstruct the entire
space-time evolution of the event.
Each entry (collision, decay or annihilation) consists of a header line
followed by 3 to $N$ lines (three lines for annihilations/decays, four
lines for scattering, possibly more lines for string-decays) 
with the individual particle information.

The event-header consists of a single line of the format:\\
\centerline{\tt format(i8,i8,i4,i7,f8.3,4e12.4)}
The format is identical to the header line for the respective 
binary interactions and decays which follow in the file.
In order to distinguish the beginning of an event from the beginning
of a collision/decay entry, the first integer in the event header is
a {\tt -1}. It is then followed by the number of the event, the mass
of projectile and target, the impact parameter, the two-particle
c.m. energy of the heavy-ion reaction (i.e. $\sqrt{s}$ for proton-proton
reactions), the total cross section of the heavy-ion reaction and
the beam energy and momentum (per particle) in the laboratory frame.

Figure \ref{collbodyfig} shows a sample collision entry. The header line
contains first the number of in- and
outgoing particles (scattering: 2 2; decay: 1 2; annihilation: 2 1;
Pauli-blocked collision: 2 0; Pauli-blocked decay: 1 0 and string decay
with 5 outgoing particles: 2 5), the ID of the respective process
(e.g. elastic scattering, decay, string excitation ...)
then the number
of the collision (in the respective event), the collision time in fm/c, 
the total CM energy ($\sqrt{s}$ in GeV), the total cross section
($\sigma_{tot}$ in mbarn), the partial cross section
($\sigma_i$ in mbarn) for the respective exit channel and
finally the baryon density at the collision point. Note that the
cross sections are ill-defined in the case of a decay.

One of the purposes of the collision file is to have the possibility to track
the trajectory of a single particle in the course of the reaction or the
time evolution of the available CM-energy per binary collision.
The contents of the particle vectors has the format:\\
\centerline{\tt format(i5,9e16.8,i11,2i3,i9,i5,i3,i15) } 
and is described in table \ref{collbodytab}. 
\begin{table}[bt]
\begin{center}
\renewcommand{\arraystretch}{1.2}
\begin{tabular}{|l|l|}\hline
\bf column\# & \bf contents \\\hline\hline
1       & $ind$ : index of particle \\
2       & $t$   : computational frame time of particle in fm/c \\
3       & $r_x$ : x coordinate in fm \\
4       & $r_y$ : y coordinate in fm \\
5       & $r_z$ : z coordinate in fm \\
6       & $E$   : energy of particle in GeV \\
7       & $p_x$ : x momentum component in GeV \\
8       & $p_y$ : y momentum component in GeV \\
9       & $p_z$ : z momentum component in GeV \\
10      & $m$ : mass of particle in GeV \\
11      & $ityp$ : particle-ID \\
12      & $2\cdot I_3$ : isospin z-projection (doubled) \\
13      & $ch$ : charge of particle  \\
14      & index of last collision partner \\
15      & $N_{coll}$ number of collisions \\
16      & $S$ : strangeness \\
17	& history information (parent process)\\\hline 
\end{tabular}
\end{center}
\caption{\label{collbodytab} contents of the particle vector in the
collision file }
\end{table}
\begin{table}[h!]
\begin{center}
\renewcommand{\arraystretch}{1.2}
\begin{tabular}{|l|l|}\hline
\bf Process ID\# & \bf Description \\\hline\hline
1	&	NN$\rightarrow$ND\\
2	&	NN$\rightarrow$NN*\\
3	&	NN$\rightarrow$ND*\\
4	&	NN$\rightarrow$DD\\
5	&	NN$\rightarrow$DN*\\
6	&	NN$\rightarrow$DD*\\
7	&	NN $\rightarrow$ N*N*,N*D*,D*D*\\
8	&	ND$\rightarrow$DD\\
10	&	MB$\rightarrow$B'\\
11	&	MM$\rightarrow$M'\\
13	&	BB (but not pp,pn) elastic scattering\\
14	&	inelastic scattering (no string excitation)\\
15	&	BB $\rightarrow$ 2 strings\\
17	&	pn-elastic\\
19	&	pp-elastic\\
20	&	decay\\
22	&	BBar elastic\\
23	&	BBar annihilation $\rightarrow$ 1 string\\
24	&	BBar diffractive $\rightarrow$ 2 strings\\
26	&	MB elastic scattering\\
27	&	MB,MM $\rightarrow$ 1 string \\
28	&	MB,MM $\rightarrow$ 2 strings\\
30	&	ND$\rightarrow$NN\\
31	&	DD$\rightarrow$DN\\
32	&	DD$\rightarrow$NN\\
35	&	ND inelastic\\
36	&	Danielewicz forward delay (MB$\rightarrow$B')\\
37	&	Danielewicz forward delay (MM$\rightarrow$M')\\
38	&	MM elastic scattering\\
39	&	BBar inelastic scattering (no annihilation)\\\hline
\end{tabular}
\end{center}
\caption{\label{processid} list of process identifiers }
\end{table}

\subsection*{Decay output: file16}

The header of the {decay output file} is identical to that of the
{\em standard output files} (see figure~\ref{headerfig} and 
table~\ref{headerformfig}).
The body of the {\em decay file} contains entries for each particle which
has decayed during the event as well as a list of all stable particles
after the final timestep of the event. 
Since the number of decays and particles listed in the event-body 
is not determined at the point where the first output is written to file,
the event body directly starts with particle-vector entries
(see tables~\ref{stdvector} and~\ref{bodytab}).
The end of the event is marked with a line of the format
{\tt format(a1,8i8)}, which contains an {\tt E} as a marker in the
first column, followed by the collision counters listed in 
table~\ref{countertab}.

For {\tt CTOption(13)=1} all outgoing particles of all collisions and
decays are listed instead of the decaying particles alone.
The {\em decay output file} provides additional
information about the produced baryon- and meson-resonances when
compared to the {\em standard output file}. Since it contains particle
output at different times during the event, however, one has to be
very careful when extracting cross sections.
All reference frame dependent values are given in the 
computational frame, which has been set by {\tt CTOption(27)}.



\begin{figure}[p]
\begin{sideways}
\tiny
\begin{minipage}[b]{\textheight}
\begin{verbatim}
UQMD   version:       10000   1000  10001  output_file  14
projectile:  (mass, char)   32  16   target:  (mass, char)   32  16 
transformation betas (NN,lab,pro)      .0000000   .7183285  -.7183285
impact_parameter_real/min/max(fm):     .00   .00   .00  total_cross_section(mbarn):        .00
equation_of_state:    0  E_lab(GeV/u):  .2000E+01  sqrt(s)(GeV):  .2697E+01  p_lab(GeV/u):  .2784E+01
event#         1 random seed:  1944955121 (fixed)  total_time(fm/c):    60 Delta(t)_O(fm/c):   60.0
op  0    0    0    0    0    0    0    0    0    0    0    0    0    0    0  
op  0    0    0    0    0    0    1    0    1    0    0    0    0    2    1  
op  0    0    0    1    1    0    0    0    0    0    0    0    0    0    0  
pa  .1000E+01    .5200E+00    .5000E+00    .3000E+00    .0000E+00    .3700E+00    .0000E+00    .9300E-01    .3500E+00    .2500E+00    .0000E+00    .5000E+00  
pa  .2700E+00    .4900E+00    .2700E+00    .1000E+01    .1600E+01    .8500E+00    .1550E+01    .0000E+00    .0000E+00    .0000E+00    .0000E+00    .0000E+00  
pa  .9000E+00    .5000E+02    .1000E+01    .1000E+01    .4000E+00    .1500E+01    .1600E+01    .0000E+00    .2500E+01    .1000E+00    .3000E+01    .2750E+00  
pa  .4200E+00    .1080E+01    .8000E+00    .5000E+00    .0000E+00    .5500E+00    .5000E+01    .8000E+00    .5000E+00    .8000E+06    .1000E+01    .2000E+01  
pvec: r0              rx              ry              rz              p0              px              py              pz              m          ityp 2i3 chg lcl#  ncl or
... start of event body ...
\end{verbatim}
\caption{\label{headerfig} sample header of an \uqmd{} output file.}
\begin{verbatim}
boxmode length(fm):   0.200000E+02 tot. energy (GeV):   0.228900E+04 s:1 p:0 #:  1
boxh ityp 2i3       N     pmax(GeV)
box:  101   1    2289  0.763000E+03
\end{verbatim}
\caption{\label{boxheaderfig} sample box-header in \uqmd{} output file.}
\begin{verbatim}
 83 60
     248     105     141       2      78     165       0       0
   .60000000E+02  -.35597252E+02  -.76801184E+01   .30121505E+01   .11839331E+01  -.70109432E+00  -.16633296E+00   .51512840E-01   .93800002E+00    1  1  1     1    9  20
   .60000000E+02   .81996126E+01   .19904670E+02  -.24104662E+02   .11707300E+01   .14606801E+00   .47854791E+00  -.49032721E+00   .93800002E+00    1 -1  0     2   17  20
   .60000000E+02   .54067910E+01   .30092770E+02   .35340946E+02   .15291829E+01   .14462310E+00   .74024320E+00   .94322867E+00   .93800002E+00    1  1  1    37    3  30
   .60000000E+02   .72953637E+00  -.35051235E+01  -.14941324E+02   .96631053E+00   .47106723E-02  -.60615381E-01  -.22408834E+00   .93800002E+00    1  1  1    13    2  19
...
\end{verbatim}
\caption{\label{eventbodyfig} beginning of a sample body of an \uqmd
standard output file.}
\begin{verbatim}
...
2       2   1      1    .398   .2650E+01   .4565E+02   .1834E+02   .1742E+01
   31   .39812730E+00   .10071119E+01   .11882873E+01   .10408219E+00   .13016664E+01  -.64792705E-01  -.13898806E-01   .91198600E+00   .92640464E+00    1 -1  0     0    0  0              0
   64   .39812730E+00   .28051872E+00   .15413296E+01   .10408219E+00   .13490995E+01   .11074780E+00   .15481079E-01  -.97761791E+00   .92294520E+00    1 -1  0     0    0  0              0
   31   .39812730E+00   .10071119E+01   .11882873E+01   .10408219E+00   .15827584E+01  -.15319538E+00   .62127758E-01  -.53196784E+00   .14814876E+01   17 -3 -1    64    1  0              1
   64   .39812730E+00   .28051872E+00   .15413296E+01   .10408219E+00   .10680074E+01   .19915048E+00  -.60545484E-01   .46633592E+00   .93800002E+00    1  1  1    31    1  0              1
2       2   5      2    .743   .2791E+01   .4448E+02   .8529E+00   .1857E+01
   25   .74262809E+00  -.11192341E+01  -.13384231E+01   .25707630E-01   .14261601E+01   .32413590E-01  -.82195855E-01   .10840402E+01   .92248724E+00    1 -1  0     0    0  0              0
   50   .74262809E+00  -.97716182E+00  -.18394695E+01   .25707630E-01   .13692160E+01  -.13287008E-01  -.56741371E-01  -.10054827E+01   .92755640E+00    1 -1  0     0    0  0              0
   25   .74262809E+00  -.11192341E+01  -.13384231E+01   .25707630E-01   .14831149E+01   .14886580E+00  -.27346802E+00  -.36638117E+00   .14030142E+01    2 -1  0    50    1  0              5
   50   .74262809E+00  -.97716182E+00  -.18394695E+01   .25707630E-01   .13122612E+01  -.12973922E+00   .13453079E+00   .44493867E+00   .12202985E+01   17 -1  0    25    1  0              5
...
1       2  20      4    .858   .1481E+01   .0000E+00   .1718E+02   .1928E+01
   31   .85784130E+00   .96261609E+00   .12063324E+01  -.50428488E-01   .15827584E+01  -.15319538E+00   .62127758E-01  -.53196784E+00   .14814876E+01   17 -3 -1    64    1  0              1
   31   .85784130E+00   .96261609E+00   .12063324E+01  -.50428488E-01   .11582730E+01  -.45590325E+00  -.19299762E+00  -.46546376E+00   .93800002E+00    1 -1  0    31    2  0             20
   65   .85784130E+00   .96261609E+00   .12063324E+01  -.50428488E-01   .42448534E+00   .30270787E+00   .25512538E+00  -.66504080E-01   .13800000E+00  101 -2 -1    31    1  0             20
...
\end{verbatim}
\caption{\label{collbodyfig} excerpts of a sample body of an \uqmd
collision history file.}
\end{minipage}
\end{sideways}
\end{figure}

\subsection*{The process identifier/parent process}

The standard output file and the collision file contain information on the current/previous sub-process.
In the collision file the information on the current process type is stored in the header of 
each individual reaction (position 3). In the standard output file entry 15 provides the ID of the 
parent process leading to the production of this particle. A list of process IDs is given in table \ref{processid}.


%\subsection*{Dilepton output: file18}
%	
%The dilepton output file contains all information necessary for
%the calculation of invariant dilepton spectra. The information
%is sufficient for spectra including density-dependent medium
%modifications.
%
%Event-header and particle entries use the same format:\\
%\centerline{\tt  format(A1,i6,4e12.4,f7.3)}
%
%As event-header the line first contains a ``{\tt !}''-character
%as flag for the beginning of an event, followed by the event-number,
%the incident beam energy, the value of {\tt CTOption(27)} (converted
%to a {\tt real*8}) and the total cross section of the heavy-ion reaction.
%
%The subsequent particle entries first contain a blank character space,
%followed by the ID-tag of the dilepton-channel (e.g. $\omega \to l^+ l^-$),
%either $dN/dM$ or $d^2N/dMd\tau$ (depending on the mode of the calculation),
%the integration step $dx$,
%the transverse momentum $p_t$ and the longitudinal momentum $p_z$ and,
%finally, the baryon-density. 

\subsection*{OSC1997A output (OSCAR 1997A format): file19}

The OSC output format has been defined by the OSCAR group in order
to create a well defined easily accessible output-format which is
supported by all OSCAR compliant transport models, event generators
and other heavy-ion related models. For a full overview of the
goals of the OSCAR collaboration, please consult the web-site\\
\centerline{\tt
\href{http://karman.physics.purdue.edu/OSCAR/}{http://karman.physics.purdue.edu/OSCAR/}
}

\uqmd{} supports the {\tt OSC1997A} output format. The file-header
consists of three lines: The first two lines have the format
{\tt format(a12)} and specify the format (e.g. {\tt OSC1997A}) and
the file contents. Currently, this is {\tt final\_id\_p\_x}, i.e. the
final event output including particle ID, the momentum space information
and the freeze-out coordinates in configuration space.
The third line has the format\\
\centerline{\tt  format (2(a8,2x),'(',i3,',',i6,')+(',i3,',',i6,')',2x,a4,2x,e10.4,2x,i8)}
and contains first the model-name and version, followed by mass and charge
of projectile and target, the reference frame of the calculation, the
incident beam energy and the number of test-particles used per nucleon.

\begin{figure}[p]
\begin{sideways}
\tiny
\begin{minipage}[b]{\textheight}
 \begin{verbatim}
OSC1997A    
final_id_p_x
   UrQMD       1.0  ( 32,    16)+( 32,    16)  eqsp   .2000E+01         1
         1          83      .000      .000
         1        2212  -.701094E+00  -.166333E+00   .515128E-01   .118393E+01   .938000E+00  -.355973E+02  -.768012E+01   .301215E+01   .600000E+02
         2        2112   .146068E+00   .478548E+00  -.490327E+00   .117073E+01   .938000E+00   .819961E+01   .199047E+02  -.241047E+02   .600000E+02
         3        2212   .144623E+00   .740243E+00   .943229E+00   .152918E+01   .938000E+00   .540679E+01   .300928E+02   .353409E+02   .600000E+02
         4        2212   .471067E-02  -.606154E-01  -.224088E+00   .966311E+00   .938000E+00   .729536E+00  -.350512E+01  -.149413E+02   .600000E+02
         5        2212  -.225070E-01  -.148103E-01   .231440E+00   .966506E+00   .938000E+00  -.866336E+00  -.961651E+00   .125389E+02   .600000E+02
         6        2212  -.241272E+00   .159999E+00  -.574162E+00   .113724E+01   .938000E+00  -.126123E+02   .657565E+01  -.298675E+02   .600000E+02
...
\end{verbatim}
\caption{\label{oscarsample} Sample output in the OSC1997A format.}
\end{minipage}
\end{sideways}
\end{figure}


The event header consists of one line with the format:\\
\centerline{\tt format(i10,2x,i10,2x,f8.3,2x,f8.3)}
This line lists the number of the event, the number of particles in
the event, the impact parameter and an azimuthal angle $\phi$ with
which the event-plane might be rotated with respect to the $xz$-plane.
The subsequent particle entries of the event-body have the format:\\
\centerline{\tt format(i10,2x,i10,2x,9(e12.6,2x))}
and contain first the particle number, its ID, then the four-momentum
vector of the particle ($p_x,p_y,p_z,E$), followed by the mass of the particle and finally its 
freeze-out location ($x_f,y_f,z_f,\tau_f$).

The particle ID is given according to the definitions of the 
{\em Review of Particle Properties} Monte-Carlo naming scheme.
Composite clusters (nuclei) are marked with {\tt 7AAAZZZ}
({\tt AAA}: mass, {\tt ZZZ}: charge of cluster). If other objects
than nuclei are used as projectile or target, then a {\tt -1} is listed
in the mass-slot followed by the PDG-ID in the charge slot.
Figure~\ref{oscarsample} shows a sample output in the OSC format.


\subsection*{OSC1999A output (OSCAR 1999A format): file20}

The {\tt OSC1999A} is an improvement to the {\tt OSC97A} output and allows
to write out the complete event history -- starting with the initial state,
including all binary collisions, string-fragmentations and hadronic decays.
In it's scope it is comparable to the \uqmd{} file15 collision output file,
but includes also the full initial configuration and final state information.
An abbreviated {\tt OSC199A} output format without the intermediate
collision history can be used as replacement for the {\tt OSC97A} output
format. The first three lines of the header are almost identical to
the {\tt OSC97A} format, but are preceded by a comment marker {\tt \#} and
a blank space in each line.

The first two lines thus  have the format
{\tt format(a20)} and specify the format (e.g. {\tt \# OSC1999A}) and
the file contents. Currently, this can be {\tt \# full\_event\_history},
the final output tag {\tt final\_id\_p\_x} referring to the OSC1997A
format. 
The third line has the format\\
\centerline{\tt  
format ('\# (',i3,',',i6,')+(',i3,',',i6,')',2x,a4,2x,e10.4,2x,i8)}
and contains first the model-name and version, followed by mass and charge
of projectile and target, the reference frame of the calculation, the
incident beam energy and the number of test-particles used per nucleon.
OSCAR further recommends that additional information be provided in
subsequent comment lines, e.g.

\begin{minipage}{10cm}
{\tt 
\#    Initial Condition:        Au + Au @ 200 GeV/c \\
\#    Cascade Time Ordering:    Center of Mass \\
\#    [additional parameters supplied for the run] \\ 
}
\end{minipage}

The remaining file after the comments contains the full history
of each event in blocks of data. Each block describes one
interaction and has the following format:\\
\begin{minipage}{10cm}
\begin{verbatim}
       block header  (one line)
       particle list (one or more lines)
\end{verbatim}
\end{minipage}

The block header contains: \\
\begin{minipage}{10cm}
\begin{verbatim}
	nin nout [optional information] 
\end{verbatim}
\end{minipage}

with {\tt nin} and {\\tt nout} being integers denoting the number of
ingoing and outgoing particles of that particular reaction -- e.g.
{\tt 2 2} for two particles scattering into two particles or
{\tt 1 2} for a resonance decaying into two particles.
Optional information can be anything that fits on one line. For example
one could put {\tt g q $\rightarrow$ g q} to characterize the block as describing
elastic quark-gluon scattering. Thus the minimum format for that line
is {\tt format (2(i7,2x))}. In \uqmd{} additional information is supplied
analogously to the file15 output: after {\tt nin} and {\tt nout}
the ID of the respective process
(e.g. elastic scattering, decay, string excitation ...) is listed,
followed by the number
of the collision (in the respective event), the collision time in fm/c, the
the total CM energy ($\sqrt{s}$ in GeV), the total cross section
($\sigma_{tot}$ in mbarn), the partial cross section
($\sigma_i$ in mbarn) for the respective exit channel and
finally the baryon density at the collision point (in units of
nuclear ground state density). Note that the
cross sections are ill-defined in the case of a decay.

The next {\tt nin} lines are incoming particles followed by
{\tt nout} lines of outgoing particles, the format of these lines is

\begin{minipage}{10cm}
\begin{verbatim}
	 ipart id ist px py pz p0 mass x y z t [optional information]  
\end{verbatim}
\end{minipage}

This format is identical to the particle entries of the OSC1997A
format with two additions: {\tt ist} is an integer containing additional 
information related to the particle ID --  this is needed in some event 
generators indicating the status of the particular entry. The optional
information can be anything useful or relevant to the particular model, but
has to fit into the same output line. 
Thus, each particle entry line
contains first the particle number, its ID, the 2nd ID-tag, 
then the four-momentum
vector of the particle ($p_x,p_y,p_z,E$), followed by the mass of the particle 
(all in GeV) and finally its production vertex [($x,y,z,\tau$); in fm and fm/c]. 
The minimal format
for this line is {\tt format (3(i10,2x),9(e12.6,2x))}.

It should be noted that the particle number {\tt ipart} is a unique
particle identifier (not equivalent to the memory slot information used
in OSC1997A) which is created for a particle at its production point and
is retired for the duration of the event at the destruction/scattering
vertex of the respective particle. It thus can be used to track trajectories
of particles in the course of the reaction.

The particle ID is given according to the definitions of the 
{\em Review of Particle Properties} Monte-Carlo naming scheme.
Composite clusters (nuclei) are marked with {\tt 7AAAZZZ}
({\tt AAA}: mass, {\tt ZZZ}: charge of cluster). If other objects
than nuclei are used as projectile or target, then a {\tt -1} is listed
in the mass-slot followed by the PDG-ID in the charge slot.

The very first block of each event describes the initial distribution
of the nucleons (partons, or other species). In this case the header 
contains
{\tt nin=0} and {\tt nout} is the number of initial particles, followed
by the respective particle vectors in the body of the block.
OSCAR recommends optional information for the event header to include
the event number, impact parameter and azimuthal angle/orientation of
that event. Thus the header of the first event block is very similar
to the OSC1997A format:

\centerline{\tt format (3(i7,2x),2(f8.3,2x))}
listing a zero, the number of initial particles, 
the number of the event,
the impact parameter and an azimuthal angle $\phi$ with
which the event-plane might be rotated with respect to the $xz$-plane.

The very last block of each event describes the final (freezeout)
configuration of all particles. Here the header comes with
{\tt nin} as the number of final particles and 
{\tt nout = 0 }.           

Figure~\ref{oscar99sample} shows a sample output in the OSC1999A format.

\begin{figure}[p]
\begin{sideways}
\tiny
\begin{minipage}[b]{\textheight}
 \begin{verbatim}
# OSC1999A
# full_event_history
# UrQMD 1.2
# (  4,     2)+(  4,     2)  nncm  0.5000E+02         1
      0        8        1     0.200     0.000
         1        2212           0  -.521296E-01  0.525298E-01  0.405082E+01  0.415387E+01  0.916548E+00  0.000000E+00  0.000000E+00  0.000000E+00  0.000000E+00
         2        2212           0  -.723310E-01  0.239309E-01  0.517619E+01  0.525948E+01  0.929208E+00  0.000000E+00  0.000000E+00  0.000000E+00  0.000000E+00
         3        2112           0  0.596589E-01  -.202471E-02  0.544394E+01  0.552250E+01  0.926260E+00  0.000000E+00  0.000000E+00  0.000000E+00  0.000000E+00
         4        2112           0  0.648017E-01  -.744360E-01  0.487618E+01  0.496485E+01  0.928902E+00  0.000000E+00  0.000000E+00  0.000000E+00  0.000000E+00
         5        2212           0  -.442186E-01  0.180513E-01  -.437103E+01  0.446873E+01  0.928092E+00  0.000000E+00  0.000000E+00  0.000000E+00  0.000000E+00
         6        2212           0  -.385991E-01  0.383077E-01  -.431142E+01  0.441015E+01  0.926336E+00  0.000000E+00  0.000000E+00  0.000000E+00  0.000000E+00
         7        2112           0  -.489669E-01  0.150743E-01  -.557163E+01  0.564798E+01  0.924068E+00  0.000000E+00  0.000000E+00  0.000000E+00  0.000000E+00
         8        2112           0  0.131784E+00  -.714333E-01  -.532665E+01  0.540741E+01  0.918942E+00  0.000000E+00  0.000000E+00  0.000000E+00  0.000000E+00
      2        4       15        1   0.118  0.9420E+01  0.3847E+02  0.2506E+02  0.1508E+01
         4        2112           0  0.648017E-01  -.744360E-01  0.487618E+01  0.496485E+01  0.928902E+00  0.669238E+00  -.290863E+00  -.383425E-01  0.117799E+00
         5        2212           0  -.442186E-01  0.180513E-01  -.437103E+01  0.446873E+01  0.928092E+00  -.205637E+00  0.215361E+00  -.383425E-01  0.117799E+00
         9        2114           0  0.376577E+00  -.473402E+00  0.223060E+01  0.261298E+01  0.121909E+01  0.669238E+00  -.290863E+00  -.383425E-01  0.117799E+00
        10        2212           0  -.311133E+00  -.221868E+00  -.318752E+01  0.334457E+01  0.938000E+00  -.205637E+00  0.215361E+00  -.383425E-01  0.117799E+00
        11       -3112           0  0.235318E+00  -.485507E+00  0.890289E+00  0.158258E+01  0.119200E+01  0.669238E+00  -.290863E+00  -.383425E-01  0.117799E+00
        12        3114           0  -.280179E+00  0.112439E+01  0.571773E+00  0.189345E+01  0.138400E+01  0.669238E+00  -.290863E+00  -.383425E-01  0.117799E+00
...
      2        2       19        3   0.290  0.9240E+01  0.3840E+02  0.7584E+01  0.1680E+01
        14        2112           0  -.425527E-01  0.836121E-01  0.366816E+01  0.378736E+01  0.938000E+00  -.101003E+01  -.746785E-01  0.898854E-01  0.290019E+00
         7        2112           0  -.489669E-01  0.150743E-01  -.557163E+01  0.564798E+01  0.924068E+00  -.888384E+00  0.244736E+00  0.869713E-01  0.290019E+00
        18        2112           0  0.845839E-02  -.167264E+00  0.366215E+01  0.378408E+01  0.938000E+00  -.101003E+01  -.746785E-01  0.898854E-01  0.290019E+00
        19        2112           0  -.999780E-01  0.265951E+00  -.556562E+01  0.565126E+01  0.938000E+00  -.888384E+00  0.244736E+00  0.869713E-01  0.290019E+00
...
      1        2       20        7   1.257  0.1687E+01  0.0000E+00  0.5665-265  0.7609E+00
        26        1212           0  -.163056E+00  -.115806E+00  -.522476E+01  0.549408E+01  0.168722E+01  0.516529E+00  -.837827E+00  -.977526E+00  0.125673E+01
        32        2214           0  -.352471E+00  -.203853E-01  -.320625E+01  0.345977E+01  0.125111E+01  0.516529E+00  -.837827E+00  -.977526E+00  0.125673E+01
        33        -211           0  0.189415E+00  -.954203E-01  -.201851E+01  0.203431E+01  0.138000E+00  0.516529E+00  -.837827E+00  -.977526E+00  0.125673E+01
...
     26        0
        39        2112           0  -.207591E+00  -.539088E-01  -.306519E+01  0.321267E+01  0.938000E+00  0.335363E+00  -.848305E+00  -.262550E+01  0.303501E+01
        28        2212           0  -.362984E+00  0.210145E+00  0.446340E+01  0.458015E+01  0.938000E+00  0.195594E+00  0.118682E+01  0.540054E-01  0.372621E+00
        35        2112           0  0.130921E+00  0.811300E-01  0.324282E+01  0.337927E+01  0.938000E+00  -.100510E+01  -.172068E+00  0.222217E+01  0.249329E+01
        46        2112           0  0.321851E+00  -.147599E+00  0.166698E+01  0.194526E+01  0.938000E+00  0.275070E+01  -.290751E+01  0.122909E+02  0.145606E+02
        21        2212           0  0.234665E+00  0.398535E+00  -.210678E+00  0.106683E+01  0.938000E+00  -.182715E+00  0.260743E+00  -.716466E-01  0.302219E+00
        29        2212           0  0.746550E+00  -.257338E+00  -.701843E+00  0.141279E+01  0.938000E+00  0.621357E+00  0.771038E+00  0.540054E-01  0.372621E+00
...
      0        0
\end{verbatim}
\caption{\label{oscar99sample} Sample output in the OSC1999A format.}
\end{minipage}
\end{sideways}
\end{figure}


\section*{Appendix A: changes from version 1.0/1.1  to 1.2}

\subsection*{New OSCAR output}

The OSC1999A OSCAR  intermediate file output format has been added.
Caution: to remain consistent with our input-file convention, you have to 
insert {\tt f20} into your urqmd-inputfile in order {\bf not} to get 
any file20 output.
If you forget this, you will get a {\tt fort.20} which will surely explode your
quota, since the output on this file is {\bf huge}.

This format is similar to our collision-file output format, but adheres
to the new OSCAR OSC1999A output convention.  Output is written to
Fortran-unit 20.

New features include particle ID according to the PDG Monte-Carlo ID scheme
and a new global quantum number, uid (stands for Unique Particle ID), i.e. a
serial number the particle gets at creation and which is retired from the
event after the particle undergoes an interaction.  This ID is supposed to
make the tracing of a particle through the collision file easier, since it
does not change dynamically as our particle-slots/numbers do, due to the
internal \uqmd{} memory management. Most file modifications are due to the
introduction of this new uid-array. As usual all output statements are found
in {\tt output.f}


\subsection*{Bug-fixes and improvements}

\begin{itemize}

\item
In {\tt GNUmakefile} there is a small modification for the code to run on 
Alpha-machines.

\item
Complete rewrite of {\tt gnuranf}. {\tt gnuranf} is now the default for Linux.

\item
A new angular distribution in the meson-baryon channel is used.
I.e. isotropic resonance decays below an inv. mass of 6~GeV and 
a forward-backward behavior above.
{\tt angdis.f}: for collisions with sroot larger than 6 GeV 
zero degree scattering is enforced (only  deflection from string decay).

\item
A sign-error in {\tt angdis.f} has been corrected. This error led to
a wrong symmetry in meson-baryon collisions (only visible when running
\uqmd{}  for elementary hadron-hadron collisions).

\item
New environment variable {\tt URQMD\_TAB} to find tables.dat
  (think of export {\tt URQMD\_TAB=tables.`uname`})

\item
Higher meltpoint for resonant meson absorption on baryons (only eta, rho,
omega and all hyperon channels).

\item
Bugfix in meson-meson annihilation cross section.

\item
{\tt string.f}:  subroutine ityp2id
case of u-quark anti-u quark:  quark ids corrected from 2 and -2  to 1 and -1.

\item
{\tt string.f}: subroutine {\tt gausspt}
comment with respect to the calculated distribution is corrected

\item
{\tt make22.f}: sighera warning only active if new logical variable  
	- {\tt warn} - is true.
          analogously defined as variables - {\tt check} - and - 
	{\tt info} - in {tt coms.f}.
          As default the variable {\tt warn} 
	is set as  false to avoid countless warnings.

\item
{\tt coms.f}:  new logical variable - {\tt warn} -
         {\tt nmax} = maximum number of particles increased from 5000 to 40000.

\item
{\tt input.f}  Due to the enhancement of nmax in coms.f a warning is added if
        calculations are performed for energies smaller 
        than 200 A GeV = $E_{lab}$ or $p_{lab}$
        or sroot smaller 20 A GeV:  parameter {\tt nmax} 
	in {\tt coms.f} may be decreased!

\item
{\tt colltab.f}:  {\tt ncollmax}: maximum number of entries in collision table
            is increased from 5000 to 10000.

\end{itemize}

\newpage

\section*{Appendix B: changes from version 1.2  to 1.3}


\subsection*{Initialization}

In \uqmd{} version 1.3 a new initialization method for cascade mode has been implemented. 
The initialization method used in \uqmd{} 1.0-1.2 led to an increased nucleon density
on the surface of the nucleus and a too small total collision crossection.  For calculations 
with Skryme equation of state ({\tt eos<>1}) the old initialization method is still the
recommended one.  
\vspace{5mm}

\syn{Caution! Due to the new initialization method calculations made with \uqmd{} version 1.3
may give results deviating from the results published with earlier versions of \uqmd!}
\vspace{5mm}

To reproduce old results you can set {\tt cto 24 0} to get the old initialization.


\subsection*{New Options}

\begin{itemize}

\item
New parameter {\tt ctp 21} allows the initialization of deformed nuclei. The
parameter 25 gives the deformation parameter (default is 0.0).

\item
New option {\tt stb [ityp]} prevents the decay of a particle type [ityp].
Multiple definitions of {\tt stb} in the input file are allowed.

\item
Phasespace correction for upper mass limit of resonances.
This feature is important for correct dilepton spectra.
To enable this feature set {\tt cto 25} to 1. Default is 0.

\item 
Alternate parametrisation of the p-pbar annihilation crosssection.
You can switch with {\tt cto 38}.
For a detailed description of the two parametrizations see Phys.Rev.C66:054903,2002.

\item
New option {\tt cto 41 2}: 

{\tt origin(i)} modified: elastic collisions no longer overwrite the production process.  
Instead elastic collisions increment the 3rd digit of origin (origin+=100).  
A process with iline=27 and no color exchange is treated as an elastic collision.

\end{itemize}
\subsection*{New Channels}

To improve the description of Kaon production at low energies the following  
channels have been implemented:

\begin{eqnarray}
p + p \Longrightarrow p + \Sigma^+ + K^0 & & \\
p + p \Longrightarrow p + \Sigma^0 + K^+ & & 
\end{eqnarray}
\begin{eqnarray}
p + p \Longrightarrow p + p + f_0 \Longrightarrow p + p+ K^+ + K^- & & \\
p + p \Longrightarrow p + p + a_0 \Longrightarrow p + p+ \bar{K}^0 + \bar{K}^- & & \\
\pi^- + p \Longrightarrow N^\ast \Longrightarrow n + f_0 \Longrightarrow n + K^+ + K^- 
\end{eqnarray}



\section*{Appendix C: changes from version 1.3  to 2.3}

UrQMD version 2.0, 2.1 and 2.2 are considered as unstable development versions.

\subsection*{Inclusion of Pythia}

PYTHIA 6.409 is included for hard scatterings from $\sqrt{s}_{\rm min} = 10 $ GeV on. Hard collisions 
are presently defined as collisions with momentum transfer $Q > 1.5$~GeV. The transition between 
the low energy string routine and Pythia is smooth and given by the probability distribution for hard scatterings. 


\begin{itemize}

\item
{\tt pythia6409.f:} This new file contains the Pythia code in version 6.409.

\item
{\tt hepnam.f,hepchg.f,hepcmp.f:} New files which convert the PDG standard Id's into useful information, such as particle names, charges and other characteristics.

\item
{\tt make22.f:} Minimal possible center of mass energy in the individual two particle reactions 
for a Pythia call is {\tt minsrt} = 10 (GeV).

\item
{\tt upmerge.f:} New file with Subroutine {\tt upyth} which merges \uqmd{} and Pythia, e.g. converts particle arrays back
and forth and finds the leading hadrons.

\item
{\tt upmerge.f:} VINT(51)=Q the momentum transfer from Pythia.

Pythia switch {\tt CTOption(44)} is implemented.

\item Leading particle cross sections for particle from PYTHIA are implemented.
{\tt leadfac} = 0.4*SUPPFAC  in {\tt upmerge.f} to reduce cross sections for leading 
hadrons out of Pythia as a simple way to account for coherence effects.

\end{itemize}

\subsection*{Adjustments of the interface PYTHIA/\uqmd}

\begin{itemize}

\item Particles unknown to \uqmd{} obtain a shift in {\tt ityp} by $\pm 1000$ (sign depends on the 
sign of the {\tt ityp}). Note that exotic PDG particle codes can now be encountered in the \uqmd{} and OSCAR output!

\item Calculate the correct charge, but for simplicity we give strangeness zero to particles produced 
via Pythia and not known to \uqmd. 
For this purpose the functions {\tt fchg(i3,i)} and {\tt strit(i)} in {\tt blockres.f} are modified.

\item 
{\tt dectim.f:} Unknown particles from Pythia are set stable ({\tt dectim(i)}=$10^{34}$).

\item
{\tt getspin.f:} Unknown particles from Pythia get spin zero.

\item
{\tt ityp2pdg.f:} Transforms \uqmd{} ityps and shifted PDG-IDs to the correct PDG-Id's.
\item
{\tt scatter.f:} In subroutine {\tt collclass}: Unknown particles from Pythia do not interact collclass=0.

\end{itemize}

\subsection*{Inclusion of high mass resonances} 
High mass resonances are included in the energy regime between $\sqrt{s}_{coll}=1.67$ GeV and 
$\sqrt{s}_{coll}=3$~GeV (upper mass limit given by {\tt CTParam(60)}). The formed particle excitations 
are treated as pseudo-resonances instead of strings. Below $\sqrt{s}_{coll}=1.67$~GeV normal resonance 
excitation via {\tt anndec} takes place. Above $\sqrt{s}_{coll}=3$~GeV the normal \uqmd-Stringroutine 
{\tt qstring} is called. Parameters for the unknown resonances are extrapolated from the nearest available resonance.
\begin{itemize}
\item 
{\tt make22.f:}
{\tt CTParam(60)} is introduced and part for {\tt iline 27} has been changed according to the above description.

\item
{\tt string.f:}
New subroutine {\tt id2itypnew} is included to obtain a  transformation from the quark ID�s to \uqmd-ityps.

\end{itemize}

To fix the strangeness production cross section which was reduced because of the new production of high mass 
resonances instead of strings the branching ratios of high lying resonances are changed to the 
corresponding branching ratios obtained from string decays of the same mass.
Further adjustments are made to keep the particle properties in line with the Particle Data Book 2006 in {\tt blockres.f}.

\subsection*{Other new features}

\begin{itemize}
\item
New Regge-parametrisations for cross-sections at high energies are implemented in {\tt make22.f}.

\item
Fix the mass distribution of the nucleon resonances $N^{*}$ via inclusion of the Delta resonances in {\tt iline 14} in {\tt make22.f:} maxnuc $\rightarrow$ maxdel.

\item  Adjust $\Xi$- and $\Omega$-production rates in pp-collisions to newly available data 
via the value of {\tt CTParam(29)}. 

\end{itemize}

\subsection*{Bug fixes}

\begin{itemize}

\item
{\tt blockres.f:} New channel (39) for baryon-antibaryon interactions to allow for string production at high energies.

\item 
{\tt coload.f:} New if-statement in {\tt ctupdate} to signal an array out of bounds error. 

\item {\tt dwidth.f:} nrejmax 5,000 $\rightarrow$ 1,000,000

(avoid warning, can be changed back if speed is more important than the details of the mass distribution)

\item
{\tt erf.f:} function erf $\rightarrow$ real*8

(declare the error function as double precision)

\item
{\tt jdecay2.f:} New variable {\tt ntry} introduced.

\item
{\tt make22.f:} ntry 100 $\rightarrow$ 1,000

\item
{\tt make22.f:} Bug in $p \bar{p}$-cross-section has been fixed. Now $pp$ and $\bar pp$ cross sections 
and scattering processes become similar at high energies. 

\item
{\tt newpart.f:} mprt 200 $\rightarrow$ 1,000

\item
{\tt string.f:} near "call getmas" 1. $\rightarrow$ 1d0

\item
{\tt urqmd.f:} Reset Pauli-blocking to old value {after final decay} for next event.

\item
{\tt urqmd.f:} Implement charge conservation check before and after the call of {\tt scatter}.
 
\item
{\tt whichres.f:} in function {\tt pcms}, lt $\rightarrow$ le
\end{itemize}

\subsection*{New formats, options and parameters}

\begin{itemize}
\item
{\tt output.f:} New subroutine {\tt urqmdlogo} is implemented to display an \uqmd{} logo which is called in {\tt init.f}.

\item
{\tt output.f:} {\bf Important: Output format has changed!!!}\\ 
Set {\tt ityp} and {\tt lstcoll} long enough for Pythia output in formats 201, 210, 213, 501, 
503: i5 $\rightarrow$ i11 and i6 $\rightarrow$ i9.\\
Time format in standard event header has been changed (line 6) to format(a7,i9,a13,i12,a9,a20,i7,a20,f11.3). 


\item 
{\tt string.f:} Single strange diquark suppression via {\tt CTParam(49)} is set to 0.5 in {\tt input.f}.

\item
{\tt input.f:}
Set {\tt CTParam(29)}= 1, as new default value $\rightarrow$ no additional double strange diquark suppression.

{\tt CTParam(59)}= 0.4, scaling factor for leading hadron cross section for Pythia particles.

{\tt CTParam(60)}= 3, resonance/string transition energy for high mass resonances

{\tt CTOption(44)}= 1, (default) call Pythia for hard scatterings 

{\tt CTOption(46)}= 0, Density calculation switch (default is baryon density) 



\item
{\tt cascinit.f:} New subroutine {\tt nucfast} if CTOption(24)=2 which provides a faster initialization $\rightarrow$ needed for cosmic air shower simulations.

\end{itemize}


\subsection*{\uqmd{} at LHC energies}
To run \uqmd{}  at LHC energies the following arrays need to be adjusted:

\begin{itemize}
\item
{\tt coms.f:} Maximum particle number {\tt nmax} should be increased from 40.000 to 100.000.

\item
{\tt colltab.f:} Size of collision table {\tt ncollmax} should be increased from 10.000 to 30.000.

\item
{\tt output.f:} The output format statements need to be changed to accomodate a larger amount of significant
digits. The following format statements need to be changed:\\
\begin{verbatim}
c standard particle information vector
 201  format(9e16.8,i11,2i3,i9,i5,i4)
LHC--> 201  format(9e24.16,i11,2i3,i9,i5,i4)

c special output for cto40 (restart of old event)
 210  format(9e16.8,i11,2i3,i9,i5,i10,3e16.8,i8)
LHC--> 210  format(9e24.16,i11,2i3,i9,i5,i10,3e24.16,i8)

c special output for mmaker
 203  format(9e16.8,i5,2i3,i6,i5,i4,i5,2e16.8)
LHC--> 203  format(9e24.16,i5,2i3,i6,i5,i4,i5,2e24.16)

c same with index for file15
 501  format(i5,9e16.8,i11,2i3,i9,i5,i3,i15)
LHC--> 501  format(i5,9e24.16,i11,2i3,i9,i5,i3,i15)

c enhanced file16
 503  format(9e15.7,i11,2i3,i9,i5,i4,2i4)
LHC--> 503  format(9e24.16,i11,2i3,i9,i5,i4,2i4)

c same including freeze-out coordinates
 213  format(9e16.8,i11,2i3,i9,i5,i4,8e16.8)
LHC--> 213  format(9e24.16,i11,2i3,i9,i5,i4,8e24.16)
\end{verbatim}

\end{itemize}


\section*{Appendix D: Patch to version 2.3}
A minor bug in the angular distribution of particles that are produced in string fragmentation (not via Pythia) has been fixed. This bug was not present in the previous published version 1.3. It has led to outgoing particles which have zero momentum in x- and y-direction in elementary p-p collisions. The multiplicities and particle spectra are unchanged by this bugfix. Thanks to Katarzyna Grebieszkow for pointing us to the problem. The following changes have been made

\begin{itemize}
\item {\tt make22.f:} New variable pythflag indicates if the process has been handled via Pythia. 
\item {\tt angdis.f:} The produced particles from \uqmd strings have to be rotated afterwards, while this is not necessary for Pythia strings. 

\end{itemize}

A rewritten GNUmakefile has been added including the directory mk with the specifications for the different running platform. gfortran is now used as the standard Linux compiler. The name of the executable has changed, please have a look in the beginning of this guide for an example file how to run the code. 

Therefore, some adjustments have been made

\begin{itemize}
\item {\tt make22.f:} go to statement in iline 27 has been removed.
\item {\tt tabinit.f}, {\tt getmass.f:} pause statements have been removed
\item {\tt upmerge.f:} the variable mm\_to\_fmc is now declared explicitly as real*8
\end{itemize}

\section*{Appendix E: changes from version 2.3 to 3.3}
\subsection*{Charm rescattering}
Implementation of charmed hadrons with the following itype's:
D (133), D* (134), J/$\Psi$ (135), $\Psi'$ (136), $\chi_c$ (137)

Rescattering cross sections with pions and rho's included as well, both elastic
and inelastic $D+\pi \leftrightarrow D^*$, $\rho + J/\Psi \leftrightarrow D+\bar{D}$ and $\rho + J/\Psi \leftrightarrow D^*+ \bar{D^*}$

Cross sections have been parameterized from work done by Zi-Wei Lin:
Nucl.Phys.A689:965-979,2001 and Phys.Rev.C62:034903,2000

\subsection*{UrQMD + Hydro}
It is possible to run UrQMD with a hydrodynamic evolution for the hot and dense stage of the heavy ion reaction. Default calculations are still the cascade mode calculations. For the physics changes please refer to arXiv:0806.1695. The hydrodynamic evolution is calculated via the SHASTA algorithm. 

\begin{itemize}
\item New files {\tt 1fluid.f}, {\tt bessel.f}, {\tt defs.f}, {\tt uhmerge.f} and new directory with tables for the equation of state (eosfiles) have been added

\item {\tt output.f}: new entry f15outhy is implemented to generate output in f15 if hydro is called. $nin$ is set to 9 and one header line and nine particle lines at the beginning and in the end of the hydrodynamic evolution is printed consisting only of zeroes except of the time information.

\item New options and parameters:
\begin{enumerate}
\item {\tt CTOption(45)}=1 : hydro mode (default is cascade calculation)
\item {\tt CTOption(47)}=2 : hadron gas EoS (default)
\item {\tt CTOption(47)}=3 : Bag model EoS
\item {\tt CTOption(47)}=5 : chiral + hadron gas EoS 
\item {\tt CTOption(48)}=N : flag for only  N timesteps of hydro evolution (test case)
\item {\tt CTOption(49)}: spectator switch: 0 (default)$\rightarrow$spectators are propagated seperately; 1 $\rightarrow$ spectators are also put on the hydro grid
\item {\tt CTOption(50)}=1: (additional) f14/f19-output directly after hydro evolution; time is equal to $t_{\rm hydrostart}$ because of back propagation, resonances decay immediately
\item {\tt CTOption(52)}: freeze-out switch: 0 (default) $\rightarrow$ isochronous transverse slices; 1 $\rightarrow$ completely isochronous freeze-out os the whole system 
\item {\tt CTOption(53)}: switch for improved momentum generation, default is zero and any other number leads to old prescription with in any case high enough maxima
\item {\tt CTParam(61)}= 0.2 fm : dx is the cell size for the hydro code
\item {\tt CTParam(62)}= 200 : ngr is the grid size of the hydro code
\item {\tt CTParam(63)}= 1. fm : is the minimal $t_{\rm hydrostart}$
\item {\tt CTParam(64)}= 5 is the factor for the freezeout criterium ($x*\epsilon_0$)
\item {\tt CTParam(65)}= 1 is multiplied with $t_{\rm hydrostart}$
\item {\tt CTParam(66)}=1.d10 is the rapidity cut for the matter that is put on the hydrodynamic grid, necessary for calculations at higher energies than $E_{\rm}=160A~$GeV. 
\end{enumerate} 

\item 
Output in timesteps according to tim statement in inputfile is not consistently possible during the hydrodynmic evolution 

\item 
The option for test cases cto 48 does not work when using the bag model equation of state. 

\item 
Cut in uhmerge.f to stabilize the Cooper-Frye Monte Carlo has been introduced. 

\item
f15-output has been adjusted. There is now one collision entry before the hydro evolution with {\tt npart} ingoing particles and no outgoing particles and the opposite after the hydro evolution. Therfore the format 502 of the header line has been changed from {\tt format(i1,i8,i4,i7,f8.3,4e12.4)} to {\tt format(i8,i8,i4,i7,f8.3,4e12.4)}. Since there is now an interaction with 0 ingoing particles which was the signal for a new event, the header-line of a new event starts now with a {\tt -1}.   


\end{itemize}

\subsection*{Bug Fixes}
\begin{itemize}
\item
Completely new makefile is written. Please use "make help" for information.
\item
{\tt anndec.f}: New subroutine {\tt getbran} which gives reasonable values back even if summed cross section is very small. 
\item
{\tt scatter.f}: Disable elastic scattering for pp collisions now works stable ({\tt CTOption(7)})
\item
{\tt blockres.f} Branching ratios for hyperon resonances are adjusted in order to ensure that they sum up to one (thanks to Pasi Huovinen).
\item 
{\tt CTParam(67)} allows for testparticle calculations (default is one testparticle per real particle). If this parameter is used with a value different from one the variable {\tt ncollmax} in {\tt colltab.f} has to increased (by CTParam(67)/2) and {\tt AAmax} should be set to 300*CTParam(67) in {\tt inputs.f}. After that the code has to be recompiled and the file 'tables.dat' has to be removed and newly generated. The output does not account for the testparticles and has to be scaled accordingly. Furthermore, the computig time increases when using this parameter. 
\item
{\tt scatter.f}: The freeze-out coordinates in position space are changed to take into account the formation times of particles produced in string fragmentation processes. Only formed hadrons are able to decouple from the system. 

\end{itemize}

\subsection{Changes for u3.3p1}

Bugfix in {\tt output.f}: wrong handling of charmed particles

\section*{Appendix F: Known problems and inconsistencies in \uqmd}

\begin{itemize}

\item
Meson-meson cross sections have discontinuities at the 
meltpoint to sighera cross section (1.7~GeV) . They
should better be treated in a similar way as the meson-baryons.


\item
More sophisticated treatment of coherent scattering will be important 
at very high energies.

\item
Detailed balance is violated due to string decays and other multi-particle
(n$\geq$3) decays, e.g. $\omega \rightarrow 3\pi$, for which
no inverse reactions are implemented.

\item
The frame dependence of the code (target vs. projectile vs. CMS-frame) 
leads to slightly asymmetric $\le 5\%$ distributions and different yields in
forward-backward hemispheres at RHIC.

\end{itemize}


\section*{Thanks}

We encourage all users to submit potential problems and bug reports 
to the following email address:
\href{mailto:urqmd@urqmd.org}{urqmd@urqmd.org}\,.

We would like to thank everybody who has been sending suggestions, bug reports and
ideas how to fix them.

Especially, Dr. Hajo Drescher, Dieter Heck and Tanguy Pierog (CORSIKA), Vladimir Uzhinsky, the HADES collaboration.



\end{document}
